\documentclass[floatsintext,man]{apa6}

\usepackage{amssymb,amsmath}
\usepackage{ifxetex,ifluatex}
\usepackage{fixltx2e} % provides \textsubscript
\ifnum 0\ifxetex 1\fi\ifluatex 1\fi=0 % if pdftex
  \usepackage[T1]{fontenc}
  \usepackage[utf8]{inputenc}
\else % if luatex or xelatex
  \ifxetex
    \usepackage{mathspec}
    \usepackage{xltxtra,xunicode}
  \else
    \usepackage{fontspec}
  \fi
  \defaultfontfeatures{Mapping=tex-text,Scale=MatchLowercase}
  \newcommand{\euro}{€}
\fi
% use upquote if available, for straight quotes in verbatim environments
\IfFileExists{upquote.sty}{\usepackage{upquote}}{}
% use microtype if available
\IfFileExists{microtype.sty}{\usepackage{microtype}}{}

% Table formatting
\usepackage{longtable, booktabs}
\usepackage{lscape}
% \usepackage[counterclockwise]{rotating}   % Landscape page setup for large tables
\usepackage{multirow}		% Table styling
\usepackage{tabularx}		% Control Column width
\usepackage[flushleft]{threeparttable}	% Allows for three part tables with a specified notes section
\usepackage{threeparttablex}            % Lets threeparttable work with longtable

% Create new environments so endfloat can handle them
% \newenvironment{ltable}
%   {\begin{landscape}\begin{center}\begin{threeparttable}}
%   {\end{threeparttable}\end{center}\end{landscape}}

\newenvironment{lltable}
  {\begin{landscape}\begin{center}\begin{ThreePartTable}}
  {\end{ThreePartTable}\end{center}\end{landscape}}




% The following enables adjusting longtable caption width to table width
% Solution found at http://golatex.de/longtable-mit-caption-so-breit-wie-die-tabelle-t15767.html
\makeatletter
\newcommand\LastLTentrywidth{1em}
\newlength\longtablewidth
\setlength{\longtablewidth}{1in}
\newcommand\getlongtablewidth{%
 \begingroup
  \ifcsname LT@\roman{LT@tables}\endcsname
  \global\longtablewidth=0pt
  \renewcommand\LT@entry[2]{\global\advance\longtablewidth by ##2\relax\gdef\LastLTentrywidth{##2}}%
  \@nameuse{LT@\roman{LT@tables}}%
  \fi
\endgroup}


  \usepackage{graphicx}
  \makeatletter
  \def\maxwidth{\ifdim\Gin@nat@width>\linewidth\linewidth\else\Gin@nat@width\fi}
  \def\maxheight{\ifdim\Gin@nat@height>\textheight\textheight\else\Gin@nat@height\fi}
  \makeatother
  % Scale images if necessary, so that they will not overflow the page
  % margins by default, and it is still possible to overwrite the defaults
  % using explicit options in \includegraphics[width, height, ...]{}
  \setkeys{Gin}{width=\maxwidth,height=\maxheight,keepaspectratio}
\ifxetex
  \usepackage[setpagesize=false, % page size defined by xetex
              unicode=false, % unicode breaks when used with xetex
              xetex]{hyperref}
\else
  \usepackage[unicode=true]{hyperref}
\fi
\hypersetup{breaklinks=true,
            pdfauthor={},
            pdftitle={Learning to Interpret a Disjunction},
            colorlinks=true,
            citecolor=blue,
            urlcolor=blue,
            linkcolor=black,
            pdfborder={0 0 0}}
\urlstyle{same}  % don't use monospace font for urls

\setlength{\parindent}{0pt}
%\setlength{\parskip}{0pt plus 0pt minus 0pt}

\setlength{\emergencystretch}{3em}  % prevent overfull lines


% Manuscript styling
\captionsetup{font=singlespacing,justification=justified}
\usepackage{csquotes}
\usepackage{upgreek}

 % Line numbering
  \usepackage{lineno}
  \linenumbers


\usepackage{tikz} % Variable definition to generate author note

% fix for \tightlist problem in pandoc 1.14
\providecommand{\tightlist}{%
  \setlength{\itemsep}{0pt}\setlength{\parskip}{0pt}}

% Essential manuscript parts
  \title{Learning to Interpret a Disjunction}

  \shorttitle{Learning Disjunction}


  \author{Masoud Jasbi\textsuperscript{1}, Akshay Jaggi\textsuperscript{2}, \& Michael C. Frank\textsuperscript{2}}

  % \def\affdep{{"", "", ""}}%
  % \def\affcity{{"", "", ""}}%

  \affiliation{
    \vspace{0.5cm}
          \textsuperscript{1} Harvard University\\
          \textsuperscript{2} Stanford University  }

  \authornote{
    Add complete departmental affiliations for each author here. Each new
    line herein must be indented, like this line.
    
    Enter author note here.
    
    Correspondence concerning this article should be addressed to Masoud
    Jasbi, Postal address. E-mail:
    \href{mailto:masoud_jasbi@fas.harvard.edu}{\nolinkurl{masoud\_jasbi@fas.harvard.edu}}
  }


  \abstract{At first glance, children's word learning appears to be mostly a problem
of learning words like \emph{dog} and \emph{run}. However, it is small
words like \emph{and} and \emph{or} that enable the construction of
complex combinatorial language. How do children learn the meaning of
these function words? Using transcripts of parent-child interactions, we
investigate the cues in child-directed speech that can inform the
interpretation and acquisition of the connective \emph{or} which has a
particularly challenging semantics. Study 1 finds that, despite its low
overall frequency, children can use \emph{or} close to parents' rate by
age 4, in some speech acts. Study 2 uses annotations of a subset of
parent-child interactions to show that disjunctions in child-directed
speech are accompanied by reliable cues to the correct interpretation
(exclusive vs.~inclusive). We present a decision-tree model that learns
from a handful of annotated examples to correctly predict the
interpretation of a disjunction. These studies suggest that conceptual
and prosodic cues in child-directed speech can provide information for
the acquisition of functional categories like disjunction.}
  \keywords{keywords \\

    \indent Word count: X
  }





\usepackage{amsthm}
\newtheorem{theorem}{Theorem}
\newtheorem{lemma}{Lemma}
\theoremstyle{definition}
\newtheorem{definition}{Definition}
\newtheorem{corollary}{Corollary}
\newtheorem{proposition}{Proposition}
\theoremstyle{definition}
\newtheorem{example}{Example}
\theoremstyle{definition}
\newtheorem{exercise}{Exercise}
\theoremstyle{remark}
\newtheorem*{remark}{Remark}
\newtheorem*{solution}{Solution}
\begin{document}

\maketitle

\setcounter{secnumdepth}{0}



\section{Introduction}\label{introduction}

\subsection{Previous Literature}\label{previous-literature}

Morris (2008) investigated the use of \emph{and} and \emph{or} in
child-directed speech and children's production between the ages of 2;0
and 5;0, using 240 transcriptions of audiotaped exchanges obtained in
the CHILDES database. Each connective was analyzed with respect to its
frequency, syntactic frame, meaning, and formal/informal use. With
respect to frequency, the study found that overall, \emph{and} is
approximately 12.8 times more likely to be produced than \emph{or}.
There were a total of 6,459 connective uses: \emph{and} was produced
5,994 times and \emph{or} 465 times.

As for the syntactic frames, instances of the connective use were coded
as appearing in \emph{statements} or \emph{questions}. Morris reported
that \emph{and} appeared predominantly in statements (more than 90\% of
the time) while \emph{or} was most common in questions (more than 85\%
of the time). For the meanings and uses of \emph{and} and \emph{or}, the
study reported that for both adults and children, the dominant meanings
of \emph{and} and \emph{or} were \enquote{conjunction} and
\enquote{exclusive disjunction}, respectively. This was taken to support
the confirmed core-meaning hypothesis. There was also a significant
increase in the mean number of different uses for \emph{and} and
\emph{or}. \emph{And} started with only the core conjunctive meaning at
2;0-2;6 and around the age of 3;0 to 4;0, children expanded it to two
different uses on average. At 4;6-5, children were producing three
different uses of \emph{and}. The production of \emph{or} started at
around 3;0-3;6 with the \enquote{exclusive} meaning and expanded to 1.5
uses on average by 4;6-5;0.

However, this account faces an important issue. The conjunctive
\emph{and} and temporal/explanation \emph{and}'s do not have the same
syntactic status. The former often conjoins noun phrases while the
latter two only conjoin clauses. We can have conjoined nouns phrases in
utterances with 3 or 4 words while conjoined clauses require utterances
longer than 4 words. How can a two-year-old with an average MLU of
1.5-2.5 words produce the temporal or explanation \emph{and} which
require conjoined clauses? It is possible that the increase in the
number of words can be explained by syntactic rather than semantic
development. The absence of non-conjunctive uses of \emph{and} in the
corpus data may only be a phenomenon in production and not
comprehension. Even if children understand the meaning of temporal
\emph{and}, if they cannot yet produce conjoined clauses, temporal
\emph{and} will not be observable in corpus data. This question cannot
be resolved from corpus evidence alone.

Utterances were also coded as informal or formal. Formal uses of
connectives were defined as utterances about truth values or states of
affairs. For example, a question like \enquote{does the dog have a
tennis ball and a hockey puck?} to which the child answered with
\enquote{No} was coded as formal. This is because the inquiry is about
the state of the world. However, \enquote{I'd like peanut butter and
jelly} was considered informal, presumably because it pertains to wants
and desires. The study found that there are rare cases of formal use in
parents' and children's speech. This is interpreted as evidence for the
developmental claim that the connectives' formal or logical (or
truth-conditional) interpretation is acquired later in development and
is not part of the core meaning.

\subsection{Current Study}\label{current-study}

Here we present 4 studies. The first study focuses on the distribution
of disjunction in adult-adult interactions. The second study looks at
the distribution of disjunctionin parent-child interactions. The third
study selects a sample of parent-child interactions and takes a closer
look at the interpretations of disjunction in discourse context. The
fourth study uses the annotations developed in the third study to train
a computational model that learns the interpretation of a disjunction
based on the cues that accompany it. We show that a learner that pays
attention to the interpretive cues accompanying disjunction can learn to
interpret it successfully as inclusive, exclusive, or even conjunctive.

\section{Study 1: Disjunction in adult-adult
interactions}\label{study-1-disjunction-in-adult-adult-interactions}

\section{Study 2: Disjunction in parent-child
interactions}\label{study-2-disjunction-in-parent-child-interactions}

\subsection{Methods}\label{methods}

For samples of parents' and children's speech, this study used the
online database \href{childes-db.stanford.edu}{childes-db} and its
associated R programming package \texttt{childesr} (Sanchez et al.,
2018). Childes-db is an online interface to the child language
components of \href{https://talkbank.org/}{TalkBank}, namely
\href{https://childes.talkbank.org/}{CHILDES} (MacWhinney, 2000) and
\href{https://phonbank.talkbank.org/}{PhonBank}. Two collections of
corpora were selected: English-North America and English-UK. All word
tokens were tagged for the following information: 1. The speaker role
(mother, father, child), 2. the age of the child when the word was
produced, 3. the type of the utterance the word appeared in
(declarative, question, imperative, other), and 4. whether the word was
\emph{and}, \emph{or}, or neither.

\begin{figure}[tb]

{\centering \includegraphics{figs/corpusDensityPlot-1} 

}

\caption{Frequency for all the words in the North America and UK corpora of CHILDES.}\label{fig:corpusDensityPlot}
\end{figure}

\subsubsection{Exclusion Criteria}\label{exclusion-criteria}

First, observations (tokens) that were coded as unintelligible were
excluded (N = 290,119). Second, observations that had missing
information on children's age were excluded (N = 1,042,478). Third,
observations outside the age range of 1 to 6 years were excluded (N =
686,870). This exclusion was because we were interested in the 1 to 6
years old age range and there was not much data outside this age range
either. The collection contained the speech of 504 children and their
parents after the exclusions.

\subsubsection{Procedure}\label{procedure}

Each token was marked for the utterance type that the token appeared in.
This study grouped utterance types into four main categories:
\enquote{declarative}, \enquote{question}, \enquote{imperative}, and
\enquote{other}. Utterance type categorization followed the convention
used in the
\href{https://talkbank.org/manuals/CHAT.html\#_Toc486414422}{TalkBank
manual}. The utterance types are similar to sentence types (declarative,
interrogative, imperative) with one exception: the category
\enquote{question} consists of interrogatives as well as rising
declaratives (i.e.~declaratives with rising question intonation). In the
transcripts, declaratives are marked with a period, questions with a
question mark, and imperatives with an exclamation mark. It is important
to note that the manual also provides
\href{https://talkbank.org/manuals/CHAT.html\#_Toc486414431}{terminators
for special-type utterances}. Among the special type utterances, this
study included the following in the category \enquote{questions}:
trailing off of a question, question with exclamation, interruption of a
question, and self-interrupted question. The category imperatives also
included \enquote{emphatic imperatives}. The rest of the special type
utterances such as \enquote{interruptions} and \enquote{trailing off}
were included in the category \enquote{other}.\\

\subsection{Properties of CHILDES
Corpora}\label{properties-of-childes-corpora}

In this section, I report some results on the distribution of words and
utterances among the speakers in our collection of corpora. The
collection contained 14,159,609 words. Table (\ref{tab:countTable})
shows the total number of \emph{and}'s, \emph{or}'s, and words in the
speech of children, fathers, and mothers. The collection contains 8.80
times more words for mothers compared to fathers and 1.80 more words for
mothers compared to children. Therefore, the collection is more
representative of the mother-child interactions than father-child
interactions. Compared to \emph{or}, the word \emph{and} is 10.80 times
more likely in the speech of mothers, 9.20 times more likely in the
speech of fathers, and 30.30 times more likely in the speech of
children. Overall, \emph{and} is 13.35 times more likely than \emph{or}
in this collection which is close to the rate reported by Morris (2008)
who used a smaller subset of CHILDES. He extracted 5,994 instances of
\emph{and} and 465 instances of \emph{or} and found that overall,
\emph{and} was 12.89 times more frequent than \emph{or} in parent-child
interactions.

\begin{table}

\caption{\label{tab:countTable}Number of \textit{and}'s, \textit{or}'s, and the total number of words in the speech of children and their parents in English-North America and English-UK collections after exclusions.}
\centering
\begin{tabular}[t]{l|r|r|r}
\hline
Speaker Role & and & or & total\\
\hline
Father & 15,488 & 1,683 & 967,075\\
\hline
Mother & 153,781 & 14,288 & 8,511,478\\
\hline
Target\_Child & 78,443 & 2,590 & 4,681,056\\
\hline
\end{tabular}
\end{table}

Figure \ref{fig:wordsByAge} shows the number of words spoken by parents
and children at each month of the child's development. The words in the
collection are not distributed uniformly and there is a high
concentration of data between the ages of 20 and 40 months (around 2 to
3 years of age). There is also a high concentration around 60 months (5
years of age). The speech of fathers shows a relatively low word-count
across all ages. Therefore, in our analyses we should be more cautious
in drawing conclusions about the speech of fathers generally, and the
speech of mothers and children after age 5. The distribution of function
words is sensitive to the type of utterance or more broadly the type of
speech act produced by speakers. For example, it is not surprising to
hear a parent say \enquote{go to your room} but a child saying the same
to a parent is unexpected. If a function word commonly occurs in such
speech acts, it is unlikely to be produced by children, even though they
may understand it very well. Therefore, it is important to check the
distribution of speech acts in corpora when studying different function
words. Since it is hard to classify and quantify speech acts
automatically, here I use utterance type as a proxy for speech acts. I
investigate the distribution of declaratives, questions, and imperatives
in this collection of corpora on parent-child interactions. Figure
\ref{fig:totalUtteranceTypePlot} shows the distribution of different
utterance types in the speech of parents and children. Overall, most
utterances are either declaratives or questions, and there are more
declaratives than questions in this collection. While mothers and
fathers show similar proportions of declaratives and questions in their
speech, children produce a lower proportion of questions and higher
proportion of declaratives than their parents.

\begin{figure}[tb]

{\centering \includegraphics{figs/totalUtteranceTypePlot-1} 

}

\caption{The proportion of declaratives and questions in children's and parents' utterances.}\label{fig:totalUtteranceTypePlot}
\end{figure}

Figure \ref{fig:utteranceTypeByAgePlot} shows the developmental trend of
declaratives and questions between the ages of one and six. Children
start with only producing declaratives and add non-declarative
utterances to their repertoire gradually until they get closer to the
parents' rate around the age six. They also start with very few
questions and increase the number of questions they ask gradually. It is
important to note that the rates of declaratives and questions in
children's speech do not reach the adult rate. These two figures show
that parent-child interactions are asymmetric. Parents ask more
questions and children produce more declaratives. This asymmetry also
interacts with age: the speech of younger children has a higher
proportion of declaratives than older children.

\begin{figure}
\centering
\includegraphics{figs/utteranceTypeByAgePlot-1.pdf}
\caption{\label{fig:utteranceTypeByAgePlot}Proportion of declaratives to
questions in parent-child interactions by age.}
\end{figure}

The frequency of function words such as \emph{and} and \emph{or} may be
affected by such conversational asymmetries if they are more likely to
appear in some utterance types than others. Figure
\ref{fig:CnctPropbySpeechAct} shows the proportion of
\emph{and}\enquote{s and \emph{or}'s that appear in different utterance
types in parents} and children's speech. In parents' speech, \emph{and}
appears more often in declaratives (around 60\% in declaratives and 20\%
in questions). On the other hand, \emph{or} appears more often in
questions than declaratives, although this difference is small in
mothers. In children's speech, both \emph{and} and \emph{or} appear most
often in declaratives. However, children have a higher proportion of
\emph{or} in questions than \emph{and} in questions.

The differences in the distribution of utterance types can affect our
interpretation of the corpus data on function words such as \emph{and}
and \emph{or} in three ways. First, since the collection contains more
declaratives than questions, it may reflect the frequency and diversity
of function words like \emph{and} that appear in declaratives better.
Second, since children produce more declaratives and fewer questions
than parents, we may underestimate children's knowledge of function
words like \emph{or} that are frequent in questions. Third, given that
the percentage of questions in the speech of children increases as they
get older, function words like \emph{or} that are more likely to appear
in questions may appear infrequent in the early stages and more frequent
in the later stages of children's development. In other words, function
words like \emph{or} that are common in questions may show a seeming
delay in production which is possibly due to the development of
questions in children's speech. Therefore, in studying children's
productions of function words, it is important to look at their relative
frequencies in different utterance types as well as the overall trends.
This is the approach I pursue in the next section.

\begin{figure}[tb]

{\centering \includegraphics{figs/CnctPropbySpeechAct-1} 

}

\caption{The proportion of \textit{and} and \textit{or} in different utterance types in the speech of parents and children.}\label{fig:CnctPropbySpeechAct}
\end{figure}

\subsection{Results}\label{study1results}

First, I consider the overall distribution of \emph{and} and \emph{or}
in the corpora and then look closer at their distributions in different
utterance types. Figure \ref{fig:freqTableBySpeakerPlot} shows the
frequency of \emph{and} and \emph{or} relative to the total number of
words produced by each speaker (i.e.~fathers, mothers, and children).
The y-axes show relative frequency per thousand words. It is also
important to note that the y-axes show different ranges of values for
\emph{and} vs. \emph{or}. This is due to the large difference between
the relative frequencies of these connectives. Overall, \emph{and}
occurs around 15 times per thousand words but \emph{or} only occurs 3
times per 2000 words in the speech of parents and around 1 time every
2000 words in the speech of children. Comparing the relative frequency
of the connectives in parents' and children's speech, we can see that
overall, children and parents produce similar rates of \emph{and} in
their interactions. However, children produce fewer \emph{or}'s than
their parents.

\begin{figure}[tb]

{\centering \includegraphics{figs/freqTableBySpeakerPlot-1} 

}

\caption{The relative frequency of \textit{and/or} in the speech of fathers, mothers, and children. 95\% binomial proportion confidence intervals calculated using Agresti-Coull's approximate method.}\label{fig:freqTableBySpeakerPlot}
\end{figure}

Next we look at the relative frequencies of \emph{and} and \emph{or} in
parents and children's speech during the course of children's
development. Figure \ref{fig:agePlot} shows the relative frequencies of
\emph{and} and \emph{or} in parents' and children's speech between 12
and 72 months (1-6 years). Production of \emph{and} in parents' speech
seems to be relatively stable and somewhere between 10 to 20
\emph{and}\enquote{s per thousand words over the course of children's
development. For children, they start producing \emph{and} between 12
and 24 months, and show a sharp increase in their production until they
reach the parent level between 30 to 36 months of age. Children stay
close to the parents} production level between 36 and 72 months,
possibly surpassing them a bit at 60 months -- although as stated in the
previous section, we should be cautious about patterns after 60 months
due to the small amount of data in this period. For \emph{or}, parents
produce between 1 to 2 \emph{or}'s every thousand words and mothers show
a slight increase in their productions between 12 to 36 months. Children
start producing \emph{or} between 18 to 30 months of age. They show a
steady increase in their productions of \emph{or} until they get close
to 1 \emph{or} per thousand words at 48 months (4 years) and stay at
that level until 72 months (6 years).

Children's productions of \emph{and} and \emph{or} show two main
differences. First, the onset of \emph{or} production is later than that
of \emph{and}. Children start producing \emph{and} around 1 to 1.5 years
old while \emph{or} productions start around 6 months later. Second,
children's \emph{and} production shows a steep rise and reaches the
parent level of production at three-years old. For \emph{or}, however,
the rise in children's production level does not reach the parent level
even though it seems to reach a constant level between the ages of 4 and
6 years.

Not reaching the parent level of \emph{or} production does not
necessarily mean that children's understanding of \emph{or} has not
fully developed yet. It can also be due to the nature of parent-child
interactions. For example, since parents ask more questions than
children and \emph{or} appears frequently in questions, parents may have
a higher frequency of \emph{or}. There are two ways of controlling for
this possibility. One is to research children's speech to peers.
Unfortunately such a large database of children's speech to peers is not
currently available for analysis. Alternatively, we can look at the
relative frequencies and developmental trends within utterance types
such as declaratives and questions to see if we spot different
developmental trends. This is what I pursue next.

\begin{figure}[tb]

{\centering \includegraphics{figs/agePlot-1} 

}

\caption{The monthly relative frequency of \textit{and/or} in parents and children's speech between 12 and 72 months (1-6 years).}\label{fig:agePlot}
\end{figure}

Figure \ref{fig:freqTablebySpeechAct} shows the relative frequency of
\emph{and} and \emph{or} in declaratives, questions, and imperatives.
\emph{And} has the highest relative frequency in declaratives while
\emph{or} has the highest relative frequency in questions. Figure
\ref{fig:ageSpeechActPlot} shows the developmental trends of the
relative frequencies of \emph{and} and \emph{or} in questions and
declaratives. Comparing \emph{and} in declaratives and questions, we see
that the onset of \emph{and} productions are slightly delayed for
questions but in both declaratives and questions, \emph{and} productions
reach the parent level around 36 months (3 years). For \emph{or}, we see
a similar delay in questions compared to declaratives. Children start
producing \emph{or} in declaratives at around 18 months but they start
producing \emph{or} in questions at 24 months. Production of \emph{or}
increases in both declaratives and questions until it seems to reach a
constant rate in declaratives between 48 and 72 months. The relative
frequency of \emph{or} in questions continues to rise until 60 months.
Comparing figures \ref{fig:agePlot} and \ref{fig:ageSpeechActPlot}, we
see that children are closer to the adult rate of production in
declaratives than questions. The large difference between parents and
children's production of \emph{or} in figure \ref{fig:agePlot} may
partly be due to the development of \emph{or} in questions. Overall the
results show that children have a substantial increase in their
productions of \emph{and} and \emph{or} between 1.5 to 4 years of age.
Therefore, it is reasonable to expect that early mappings for the
meaning and usage of these words develop in this age range.

\begin{figure}[tb]

{\centering \includegraphics{figs/freqTablebySpeechAct-1} 

}

\caption{Relative frequency of \textit{and/or} in declaratives, imperatives, and interrogatives for parents and children }\label{fig:freqTablebySpeechAct}
\end{figure}

\begin{figure}[tb]

{\centering \includegraphics{figs/ageSpeechActPlot-1} 

}

\caption{Relative frequency of \textit{and/or} in declaratives and questions for parents and childern between the child-age of 12 and 72 months (1-6 years).}\label{fig:ageSpeechActPlot}
\end{figure}

\subsection{Discussion}\label{study1discussion}

The goal of this study was to explore the frequency of \emph{and} and
\emph{or} in parents and children's speech. The study found three
differences. First, it found a difference between the overall frequency
of \emph{and} and \emph{or} in both parents and children. \emph{And} was
about 10 times more frequent than \emph{or} in the speech of parents and
30 times more likely in the speech of children. Second, the study found
a difference between parents' and children's productions of \emph{or}.
Relative to the total number of words spoken by parents and children
between the ages of 1 and 6 years, both children and parents produce on
average 15 \emph{and}\enquote{s every 1000 words. Therefore, children
match parents} rate of \emph{and} production overall. This is not the
case for \emph{or} as parents produce 3 \emph{or}\enquote{s every 2000
words and children only 1 every 2000 words. Third, the study found a
developmental difference between \emph{and} and \emph{or} as well. The
study found that the onset of production is earlier for \emph{and} than
\emph{or}. In the monthly relative frequencies of \emph{and} and
\emph{or} in the speech of parents and children, the study also found
that children reach the parents} level of production for \emph{and} at
age 3 while \emph{or} does not reach the parents' level even at age 6.

What causes these production differences? The first difference -- that
\emph{and} is far more frequent than \emph{or} -- is not surprising or
limited to child-directed speech. \emph{And} is useful in a large set of
contexts from conjoining elements of a sentence to connecting discourse
elements or even holding the floor and delaying a conversational turn.
In comparison, \emph{or} seems to have a more limited usage. The second
and the third differences -- namely that children produce fewer
\emph{or}\enquote{s than parents, and that they produce \emph{and} and
reach their parents rate earlier than \emph{or} -- could be due to three
factors. First, production of \emph{and} develops and reaches the
parents} rate earlier possibly because it is much more frequent than
\emph{or} in children's input. Previous research suggests that within
the same syntactic category, words with higher frequency in
child-directed speech are acquired earlier (Goodman, Dale, \& Li, 2008).
The conjunction word \emph{and} is at least 10 times more likely than
\emph{or} so earlier acquisition of \emph{and} is consistent with the
effect of frequency on age of acquisition. Second, research on concept
attainment has suggested that the concept of conjunction is easier to
conjure and possibly acquire than the concept of disjunction. In
experiments that participants are asked to detect a pattern in the
classification of cards, participants can detect a conjunctive
classification pattern faster than a disjunctive one (Neisser \& Weene,
1962). Therefore, it is possible that children learn the meaning of
\emph{and} faster and start to produce it earlier but they need more
time to figure out the meaning and usage of \emph{or}.

A third possibility is that the developmental difference between
\emph{and} and \emph{or} is mainly due to the asymmetric nature of
parent-child interactions and the utterance types that each role in this
interaction requires. For example, this study found that parents ask
more questions of children than children do of parents. It also found
that \emph{or} is much more frequent in questions than \emph{and} is.
Therefore, parent-child interaction provides more opportunities for
parents to use \emph{or} than children. In the next study we will
discuss several constructions and communicative functions that are also
more appropriate for the role of parents. For example, \emph{or} is
often used to ask what someone else wants like \enquote{do you want
apple juice or orange juice?} or for asking someone to clarify what they
said such as \enquote{did you mean ball or bowl?}. Both of these
constructions are more likely to be produced by a parent than a child.
\emph{Or} is also used to introduce examples or provide definitions such
as \enquote{an animal, like a rabbit, or a lion, or a sheep}. It is very
unlikely that children would use such constructions to define terms for
parents! Furthermore, such constructions also reveal their own
developmental trends. For example, the study found that children start
by almost entirely producing declaratives and increase their questions
until at age 4 to 6, about 10\% of their utterances are questions.
Therefore, children's ability to produce \emph{or} in a question is
subject to the development of questions themselves. More generally, the
developmental difference between \emph{and} and \emph{or} may also be
due to a difference in the development of other factors that production
of \emph{and} and \emph{or} rely on, such as the development of
constructions with specific communicative functions like unconditionals
(Whether X or Y, discussed in Chapter \ref{sempragLit}). In future
research, it will be important to establish the extent to which each of
these potential causes -- frequency, conceptual complexity, and the
development of other factors such as utterance type or constructions
with specific communicative functions -- contribute to the developmental
differences in the production of conjunction and disjunction.

\section{Study 3: Interpretations of disjunction in child-directed
speech}\label{study-3-interpretations-of-disjunction-in-child-directed-speech}

Previous study reported on the frequencies of disjunction in parents and
children's speech production. To help us better understand children's
linguistic input, this study offers a close examination of the
interpretations that \emph{and} and \emph{or} have in child-directed
speech. It had two main goals. First, to replicate the finding of Morris
(2008) and second, to identify any cues in children's input that might
help them learn the interpretations of disjunction in English.

\subsection{Methods}\label{methods-1}

This study used
\href{https://phonbank.talkbank.org/browser/index.php?url=Eng-NA/Providence/}{the
Providence corpus} (Demuth, Culbertson, \& Alter, 2006) available via
the \href{https://phonbank.talkbank.org}{PhonBank} section of
\href{https://talkbank.org/}{the TalkBank.org archive}. The corpus was
chosen because of its relatively dense data on child-directed speech as
well as the availability of audio and video recordings that would allow
annotators access to the context of the utterance. The corpus was
collected between 2002 and 2005 in Providence, Rhode Island. Table
\ref{tab:providence} reports the name, age range, and the number of
recording sessions for the participants in the study. All children were
monolingual English speakers and were followed between the ages of 1 and
4 years. Based on Study 2, this is the age range when children develop
their early understanding or mappings for the meanings of \emph{and} and
\emph{or}. The corpus contains roughly biweekly hour-long recordings of
spontaneous parent-child interactions, with most recordings being of
mother-child interactions. The corpus consists of a total of 364 hours
of speech.

\begin{longtable}[]{@{}ccc@{}}
\caption{\label{tab:providence} Information on the participants in the
Providence Corpus. Ethan was diagnosed with Asperger's syndrome and
therefore was excluded from this study.}\tabularnewline
\toprule
Name & Age Range & Sessions\tabularnewline
\midrule
\endfirsthead
\toprule
Name & Age Range & Sessions\tabularnewline
\midrule
\endhead
Alex & 1;04.28-3;05.16 & 51\tabularnewline
Ethan & 0;11.04-2;11.01 & 50\tabularnewline
Lily & 1;01.02-4;00.02 & 80\tabularnewline
Naima & 0;11.27-3;10.10 & 88\tabularnewline
Violet & 1;02.00-3;11.24 & 51\tabularnewline
William & 1;04.12-3;04.18 & 44\tabularnewline
\bottomrule
\end{longtable}

\subsubsection{Exclusion Criteria}\label{exclusion-criteria-1}

We excluded data from Ethan since he was diagnosed with Asperger's
Syndrome at age 5. We also excluded all examples found in conversations
over the phone, adult-adult conversations, and utterances heard from TV
or radio. We did not count such utterances as child-directed speech. We
excluded proper names and fixed forms such as \enquote{Bread and Circus}
(name of a local place) or \enquote{trick-or-treat} from the set of
examples to be annotated. The rationale here was that such forms could
be learned and understood with no actual understanding of the connective
meaning. We counted multiple instances of \emph{or} and \emph{and}
within the same disjunction/conjunction as one instance. The reasoning
was that, in a coordinated structure, the additional occurrences of a
connective typically did not alter the annotation categories, and most
importantly the interpretation of the coordination. For example, there
is almost no difference between \enquote{cat, dog, and elephant} versus
\enquote{cat and dog and elephant} in interpretation. In short, we
focused on the \enquote{coordinated construction} as a unit rather than
on every separate instance of \emph{and} and \emph{or}. Instances of
multiple connectives in a coordination were rare in the corpus.

\subsubsection{Procedure}\label{procedure-1}

All utterances containing \emph{and} and \emph{or} were extracted using
\href{http://alpha.talkbank.org/clan/}{the CLAN software} and
automatically tagged for the following: (1) the name of the child; (2)
the transcript address; (3) the speaker of the utterance (father,
mother, or child); (4) the child's birth date, and (5) the recording
date. Since the focus of the study was mainly on disjunction, we
annotated instances of \emph{or} in all the child-directed speech from
the earliest examples to the latest ones found. Given that the corpus
contained more than 10 times the number of \emph{and}'s than
\emph{or}'s, we randomly sampled 1000 examples of \emph{and} to match
1000 examples of \emph{or}. Here we report the results on 465 examples
of \emph{and} and 608 examples of \emph{or}.

\subsubsection{Annotation Categories}\label{annotation-categories}

Every extracted instance of \emph{and} and \emph{or} was manually
annotated for 7 categories: 1. Connective Interpretation 2. Intonation
Type 3. Utterance Type 4. Syntactic Level 5. Conceptual Consistency 6.
Communicative Function and 7. Answer Type. In what follows, I explain
how each annotation category was defined in detail and provide some
prototypical examples of the category.

\paragraph{Connective Interpretation}\label{connective-interpretation}

This category is the dependent variable of the study. Annotators
listened to coordinations such as \enquote{A or B} and \enquote{A and
B}, and decided the intended interpretation of the connective with
respect to the truth of A and B. We used the sixteen binary connectives
shown in Figure \ref{fig:logicalConnectives} as the space of possible
connective interpretations. Annotators were asked to consider the two
propositions raised by the coordinated construction, ignoring the
connective and functional elements such as negation and modals. Consider
the following sentences containing \emph{or}: \enquote{Bob plays soccer
or tennis} and \enquote{Bob doesn't play soccer or tennis}. Both discuss
the same two propositions: A. Bob playing soccer, and B. Bob playing
tennis. However, the functional elements combining these two
propositions result in different interpretations with respect to the
truth of A and B. In \enquote{Bob plays soccer or tennis} which contains
a disjunction, the interpretation is that Bob plays one or possibly both
sports (inclusive disjunction IOR). In \enquote{Bob doesn't play soccer
or tennis} which contains a negation and a disjunction, the
interpretation is that Bob plays neither sport (NOR). For connective
interpretations, the annotators first reconstructed the coordinated
propositions without the connectives or negation and then decided which
propositions were implied to be true/false.

This approach is partly informed by children's development of function
and content words. Since children acquire content words earlier than
functions words, we assumed that when learning logical connectives, they
better understand the content of the propositions being coordinated
rather than the functional elements involved in building the coordinated
construction. For example, considering the sentences \enquote{Bob
doesn't play soccer or tennis} without its function words as
\enquote{Bob, play, soccer, tennis}, one can still deduce that there are
two relevant propositions: Bob playing soccer, and Bob playing tennis.
However, the real challenge is to figure out what is being communicated
with respect to the truth of these two propositions. If the learner can
figure this out, then the meaning of the functional elements can be
reverse engineered. For example, if the learner recognizes that
\enquote{Bob plays soccer or tennis} communicates that one or both
propositions are true (IOR), the learner can associate this
interpretation to the unknown element \emph{or}. Similarly, if the
learner recognizes the interpretation of \enquote{Bob doesn't play
soccer or tennis} as neither proposition is true (NOR), they can
associate this interpretation to the combination of disjunction and the
overt sentential negation. Table \ref{tab:connectiveInterpretaion}
reports the connective interpretations found in our annotations as well
as some examples for each interpretation.

\begin{figure}[tb]

{\centering \includegraphics{figs/logicalConnectives-1} 

}

\caption{The truth table for the 16 binary logical connectives. The rows represent the set of situations where zero, one, or both propositions are true. The columns represent the 16 possible connectives and their truth conditions. Green cells represent true situations.}\label{fig:logicalConnectives}
\end{figure}

\begin{longtable}[]{@{}lll@{}}
\caption{\label{tab:connectiveInterpretaion} Annotation classes for
connective interpretation}\tabularnewline
\toprule
\begin{minipage}[b]{0.07\columnwidth}\raggedright\strut
Class\strut
\end{minipage} & \begin{minipage}[b]{0.36\columnwidth}\raggedright\strut
Meaning\strut
\end{minipage} & \begin{minipage}[b]{0.48\columnwidth}\raggedright\strut
Examples\strut
\end{minipage}\tabularnewline
\midrule
\endfirsthead
\toprule
\begin{minipage}[b]{0.07\columnwidth}\raggedright\strut
Class\strut
\end{minipage} & \begin{minipage}[b]{0.36\columnwidth}\raggedright\strut
Meaning\strut
\end{minipage} & \begin{minipage}[b]{0.48\columnwidth}\raggedright\strut
Examples\strut
\end{minipage}\tabularnewline
\midrule
\endhead
\begin{minipage}[t]{0.07\columnwidth}\raggedright\strut
AND\strut
\end{minipage} & \begin{minipage}[t]{0.36\columnwidth}\raggedright\strut
Both propositions are true\strut
\end{minipage} & \begin{minipage}[t]{0.48\columnwidth}\raggedright\strut
\emph{\enquote{I'm just gonna empty this and then I'll be out of the
kitchen.} -- \enquote{I'll mix them together or I could mix it with
carrot, too.}}\strut
\end{minipage}\tabularnewline
\begin{minipage}[t]{0.07\columnwidth}\raggedright\strut
IOR\strut
\end{minipage} & \begin{minipage}[t]{0.36\columnwidth}\raggedright\strut
One or both propositions are true\strut
\end{minipage} & \begin{minipage}[t]{0.48\columnwidth}\raggedright\strut
\emph{\enquote{You should use a spoon or a fork.} -- \enquote{Ask a
grownup for some juice or water or soy milk.}}\strut
\end{minipage}\tabularnewline
\begin{minipage}[t]{0.07\columnwidth}\raggedright\strut
XOR\strut
\end{minipage} & \begin{minipage}[t]{0.36\columnwidth}\raggedright\strut
Only one proposition is true\strut
\end{minipage} & \begin{minipage}[t]{0.48\columnwidth}\raggedright\strut
\emph{\enquote{Is that a hyena? or a leopard?} -- \enquote{We're gonna
do things one way or the other.}}\strut
\end{minipage}\tabularnewline
\begin{minipage}[t]{0.07\columnwidth}\raggedright\strut
NOR\strut
\end{minipage} & \begin{minipage}[t]{0.36\columnwidth}\raggedright\strut
Neither proposition is true\strut
\end{minipage} & \begin{minipage}[t]{0.48\columnwidth}\raggedright\strut
\emph{\enquote{I wouldn't say boo to one goose or three.} --
\enquote{She found she lacked talent for hiding in trees, for chirping
like crickets, or humming like bees.}}\strut
\end{minipage}\tabularnewline
\begin{minipage}[t]{0.07\columnwidth}\raggedright\strut
IFF\strut
\end{minipage} & \begin{minipage}[t]{0.36\columnwidth}\raggedright\strut
Either both propositions are true or both are false\strut
\end{minipage} & \begin{minipage}[t]{0.48\columnwidth}\raggedright\strut
\emph{\enquote{Put them {[}crayons{]} up here and you can get down. --
Come over here and I'll show you.}}\strut
\end{minipage}\tabularnewline
\begin{minipage}[t]{0.07\columnwidth}\raggedright\strut
NAB\strut
\end{minipage} & \begin{minipage}[t]{0.36\columnwidth}\raggedright\strut
The first proposition is false, the second is true.\strut
\end{minipage} & \begin{minipage}[t]{0.48\columnwidth}\raggedright\strut
\emph{\enquote{There's an Oatio here, or actually, there's a wheat
here.}}\strut
\end{minipage}\tabularnewline
\bottomrule
\end{longtable}

\paragraph{Intonation Type}\label{intonation-type}

Annotators listened to the utterances and decided whether the intonation
contour on the coordination was flat, rise, or rise-fall. Table
\ref{tab:intonationTypes} shows the definitions and examples for these
intonation types. In order to judge the intonation of the sentence
accurately, annotators were asked to construct all three intonation
contours for the sentence and see which one is closer to the actual
intonation of the utterance. For example, to judge the sentence
\enquote{do you want orange juice\(\uparrow\) or apple
juice\(\downarrow\)?}, they reconstructed the sentence with the
prototypical flat, rising, and rise-fall intonations and checked to see
which intonation is closer to the actual one. It is important to note
that while these three intonation contours provide a good general
classification, there is a substantial degree of variation as well as a
good number of subtypes within each intonation type.

\begin{longtable}[]{@{}lll@{}}
\caption{\label{tab:intonationTypes} Definitions of the intonation types and
their examples.}\tabularnewline
\toprule
\begin{minipage}[b]{0.11\columnwidth}\raggedright\strut
Intonation\strut
\end{minipage} & \begin{minipage}[b]{0.44\columnwidth}\raggedright\strut
Definitions\strut
\end{minipage} & \begin{minipage}[b]{0.34\columnwidth}\raggedright\strut
Examples\strut
\end{minipage}\tabularnewline
\midrule
\endfirsthead
\toprule
\begin{minipage}[b]{0.11\columnwidth}\raggedright\strut
Intonation\strut
\end{minipage} & \begin{minipage}[b]{0.44\columnwidth}\raggedright\strut
Definitions\strut
\end{minipage} & \begin{minipage}[b]{0.34\columnwidth}\raggedright\strut
Examples\strut
\end{minipage}\tabularnewline
\midrule
\endhead
\begin{minipage}[t]{0.11\columnwidth}\raggedright\strut
Flat\strut
\end{minipage} & \begin{minipage}[t]{0.44\columnwidth}\raggedright\strut
Intonation does not show any substantial rise at the end of the
sentence.\strut
\end{minipage} & \begin{minipage}[t]{0.34\columnwidth}\raggedright\strut
\emph{\enquote{I don't hear any meows or bow-wow-wows.}}\strut
\end{minipage}\tabularnewline
\begin{minipage}[t]{0.11\columnwidth}\raggedright\strut
Rise\strut
\end{minipage} & \begin{minipage}[t]{0.44\columnwidth}\raggedright\strut
There is a substantial intonation rise on each disjunct or generally on
both.\strut
\end{minipage} & \begin{minipage}[t]{0.34\columnwidth}\raggedright\strut
\emph{\enquote{Do you want some seaweed? or some wheat germ?}}\strut
\end{minipage}\tabularnewline
\begin{minipage}[t]{0.11\columnwidth}\raggedright\strut
Rise-Fall\strut
\end{minipage} & \begin{minipage}[t]{0.44\columnwidth}\raggedright\strut
There is a substantial rise on the non-final disjunct(s), and a fall on
the final disjunct.\strut
\end{minipage} & \begin{minipage}[t]{0.34\columnwidth}\raggedright\strut
\emph{\enquote{Is that big Q or little q?} -- \enquote{(are) You patting
them, petting them, or slapping them?}}\strut
\end{minipage}\tabularnewline
\bottomrule
\end{longtable}

\paragraph{Utterance Type}\label{utterance-type}

Annotators decided whether an utterance is a declarative, an
interrogative, or an imperative. Table \ref{tab:utteranceTypes} provide
the definitions and examples for each utterance type. Occasionally, we
found examples with different utterance types for each coordinand. For
example, the mother would say \enquote{put your backpack on and I'll be
right back}, where the first cooridnand is an imperative and the second
a declarative. Such examples were coded for both utterance types with a
dash in-between: imperative-declarative.

\begin{longtable}[]{@{}lll@{}}
\caption{\label{tab:utteranceTypes} Definitions of the utterance types and
their examples.}\tabularnewline
\toprule
\begin{minipage}[b]{0.18\columnwidth}\raggedright\strut
Utterance Types\strut
\end{minipage} & \begin{minipage}[b]{0.42\columnwidth}\raggedright\strut
Definitions\strut
\end{minipage} & \begin{minipage}[b]{0.32\columnwidth}\raggedright\strut
Examples\strut
\end{minipage}\tabularnewline
\midrule
\endfirsthead
\toprule
\begin{minipage}[b]{0.18\columnwidth}\raggedright\strut
Utterance Types\strut
\end{minipage} & \begin{minipage}[b]{0.42\columnwidth}\raggedright\strut
Definitions\strut
\end{minipage} & \begin{minipage}[b]{0.32\columnwidth}\raggedright\strut
Examples\strut
\end{minipage}\tabularnewline
\midrule
\endhead
\begin{minipage}[t]{0.18\columnwidth}\raggedright\strut
Declarative\strut
\end{minipage} & \begin{minipage}[t]{0.42\columnwidth}\raggedright\strut
A statement with a subject-verb-object word order and a flat
intonation.\strut
\end{minipage} & \begin{minipage}[t]{0.32\columnwidth}\raggedright\strut
\emph{\enquote{It looks a little bit like a drum stick or a
mallet.}}\strut
\end{minipage}\tabularnewline
\begin{minipage}[t]{0.18\columnwidth}\raggedright\strut
Interrogative\strut
\end{minipage} & \begin{minipage}[t]{0.42\columnwidth}\raggedright\strut
A question with either subject-auxiliary inversion or a rising terminal
intonation.\strut
\end{minipage} & \begin{minipage}[t]{0.32\columnwidth}\raggedright\strut
\emph{\enquote{Is that a dog or a cat?}}\strut
\end{minipage}\tabularnewline
\begin{minipage}[t]{0.18\columnwidth}\raggedright\strut
Imperative\strut
\end{minipage} & \begin{minipage}[t]{0.42\columnwidth}\raggedright\strut
A directive with an uninflected verb and no subject\strut
\end{minipage} & \begin{minipage}[t]{0.32\columnwidth}\raggedright\strut
\emph{\enquote{Have a little more French toast or have some of your
juice.}}\strut
\end{minipage}\tabularnewline
\bottomrule
\end{longtable}

\paragraph{Syntactic Level}\label{syntactic-level}

For this annotation category, annotators decided whether the
coordination is at the clausal level or at the sub-clausal level.
Clausal level was defined as sentences, clauses, verb phrases, and
verbs. Coordination of other categories was coded as sub-clausal. This
annotation category was introduced to check the hypothesis that the
syntactic category of the coordinands may influence the interpretation
of a coordination. The intuition was that a sentence such as \enquote{He
drank tea or coffee} is less likely to be interpreted as exclusive than
\enquote{He drank tea or he drank coffee.} The clausal vs.~sub-clausal
distinction was inspired by the fact that in many languages,
coordinators that connect sentences and verb phrases are different
lexical items than those that connect nominal, adjectival, or
prepositional phrases (see Haspelmath, 2007).

\begin{longtable}[]{@{}lll@{}}
\caption{\label{tab:syntacticLevel} Definitions of the syntactic levels and
their examples.}\tabularnewline
\toprule
\begin{minipage}[b]{0.17\columnwidth}\raggedright\strut
Syntactic Level\strut
\end{minipage} & \begin{minipage}[b]{0.37\columnwidth}\raggedright\strut
Definitions\strut
\end{minipage} & \begin{minipage}[b]{0.37\columnwidth}\raggedright\strut
Examples\strut
\end{minipage}\tabularnewline
\midrule
\endfirsthead
\toprule
\begin{minipage}[b]{0.17\columnwidth}\raggedright\strut
Syntactic Level\strut
\end{minipage} & \begin{minipage}[b]{0.37\columnwidth}\raggedright\strut
Definitions\strut
\end{minipage} & \begin{minipage}[b]{0.37\columnwidth}\raggedright\strut
Examples\strut
\end{minipage}\tabularnewline
\midrule
\endhead
\begin{minipage}[t]{0.17\columnwidth}\raggedright\strut
Clausal\strut
\end{minipage} & \begin{minipage}[t]{0.37\columnwidth}\raggedright\strut
The coordinands are sentences, clauses, verb phrases, or verbs.\strut
\end{minipage} & \begin{minipage}[t]{0.37\columnwidth}\raggedright\strut
\emph{\enquote{Does he lose his tail sometimes and Pooh helps him and
puts it back on?}}\strut
\end{minipage}\tabularnewline
\begin{minipage}[t]{0.17\columnwidth}\raggedright\strut
Sub-clausal\strut
\end{minipage} & \begin{minipage}[t]{0.37\columnwidth}\raggedright\strut
The coordinands are nouns, adjectives, noun phrases, determiner phrases,
or prepositional phrases.\strut
\end{minipage} & \begin{minipage}[t]{0.37\columnwidth}\raggedright\strut
\emph{\enquote{Hollies can be bushes or trees.}}\strut
\end{minipage}\tabularnewline
\bottomrule
\end{longtable}

\paragraph{Conceptual Consistency}\label{conceptual-consistency}

Propositions that are connected by words such as \emph{and} and
\emph{or} often stand in complex conceptual relations with each other.
For conceptual consistency, annotators decided whether the propositions
that make up the coordination can be true at the same time or not. If
the two propositions could be true at the same time they were marked as
consistent. If the two propositions could not be true at the same time
and resulted in a contradiction, they were marked as inconsistent. Our
annotators used the following diagnostic to decide the consistency of
the disjuncts: Two disjuncts were marked as inconsistent if replacing
the word \emph{or} with \emph{and} produced a contradiction. For
example, changing \enquote{the ball is in my room \emph{or} your room}
to \enquote{the ball is in my room \emph{and} your room} produces a
contradiction because a ball cannot be in two rooms at the same time.

\begin{longtable}[]{@{}lll@{}}
\caption{\label{tab:consistencyType} Definitions of consistency types and
their examples.}\tabularnewline
\toprule
\begin{minipage}[b]{0.14\columnwidth}\raggedright\strut
Consistency\strut
\end{minipage} & \begin{minipage}[b]{0.25\columnwidth}\raggedright\strut
Definitions\strut
\end{minipage} & \begin{minipage}[b]{0.42\columnwidth}\raggedright\strut
Examples\strut
\end{minipage}\tabularnewline
\midrule
\endfirsthead
\toprule
\begin{minipage}[b]{0.14\columnwidth}\raggedright\strut
Consistency\strut
\end{minipage} & \begin{minipage}[b]{0.25\columnwidth}\raggedright\strut
Definitions\strut
\end{minipage} & \begin{minipage}[b]{0.42\columnwidth}\raggedright\strut
Examples\strut
\end{minipage}\tabularnewline
\midrule
\endhead
\begin{minipage}[t]{0.14\columnwidth}\raggedright\strut
Consistent\strut
\end{minipage} & \begin{minipage}[t]{0.25\columnwidth}\raggedright\strut
The coordinands can be true at the same time.\strut
\end{minipage} & \begin{minipage}[t]{0.42\columnwidth}\raggedright\strut
\emph{\enquote{We could spell some things with a pen or draw some
pictures.}}\strut
\end{minipage}\tabularnewline
\begin{minipage}[t]{0.14\columnwidth}\raggedright\strut
Inconsistent\strut
\end{minipage} & \begin{minipage}[t]{0.25\columnwidth}\raggedright\strut
The coordinands cannot be true at the same time.\strut
\end{minipage} & \begin{minipage}[t]{0.42\columnwidth}\raggedright\strut
\emph{\enquote{Do you want to stay or go?}}\strut
\end{minipage}\tabularnewline
\bottomrule
\end{longtable}

First, it is important to note here that this criterion is quite strict.
In many cases, the possibility of both propositions being true is ruled
out based on prior knowledge and expectations of the situation. For
example, when asking people whether they would like tea or coffee, it is
often assumed and expected that people choose one or the other. However,
wanting to drink both tea and coffee is not conceptually inconsistent.
It is just very unlikely. Our annotations of consistency are very
conservative in that they still consider such unlikely cases as
consistent. Relaxing this criterion to capture the unlikely cases may
increase exclusivity inferences that are caused by alternatives that are
considered unlikely to co-occur.

Second, there are other more complex relations between coordinated
propositions that we have not coded for. For example, coordinated
propositions sometimes stand in a causal relation (e.g.~the cup fell and
broke) or sometimes in a temporal relation (e.g.~she brushed her teeth
and went to bed), among many more. It is quite feasible to assume that
the rich conceptual structure of these propositions help children learn
the meaning and use of connectives such as \emph{and}, \emph{or},
\emph{if}, \emph{therefore}, etc. It is possible to develop a more
detailed investigation on the relation between propositions and how that
affects the acquisition of connective meaning generally. However, in
this study we mainly focus on conceptual consistency of the coordinated
propositions and how that affects the acquisition of \emph{and} and
\emph{or}.

It is also important to note that if the coordinands are inconsistent,
this does not necessarily means that the connective interpretation must
be exclusive. For example, in a sentence like \enquote{you could stay
here or go out}, the alternatives \enquote{staying here} and
\enquote{going out} are inconsistent. Yet, the overall interpretation of
the connective could be conjunctive: you could stay here AND you could
go out. The statement communicates that both possibilities hold. This
pattern of interaction between possibility modals like \emph{can} and
disjunction words like \emph{or} are often discussed under the label
\enquote{free-choice inferences} in the semantics and pragmatics
literature (Kamp, 1973; Von Wright, 1968). Another example is
unconditionals such as \enquote{Ready or not, here I come!}. The
coordinands are contradictions: one is the negation of the other.
However, the overall interpretation of the sentences is that in both
cases, the speaker is going to come.

\paragraph{Communicative Functions}\label{communicative-functions}

This study constructed a set of categories that captured particular
usages or communicative functions of the words \emph{or} and \emph{and}.
These communicative functions were created using the first 100 examples
and then they were used for the classification of the rest of the
examples. Table \ref{tab:speechActs} shows the definitions and examples
of the 10 communicative functions used in this study. The table contains
some functions that are general and some that are specific to
coordination. For example, directives are a general class while
conditionals are more specific to coordinated constructions. It is also
important to note that the list is not unstructured. Some communicative
functions are subtypes of others. For example, \enquote{identifications}
and \enquote{unconditionals} are subtypes of \enquote{descriptions}
while \enquote{conditionals} are a subtype of directives. Furthermore,
\enquote{repairs} seem parallel to other categories in that any speech
at can be repaired. We do not fully explore the details of these
functions in this study but such details matter for a general theory of
acquisition that makes use of the speaker's communicative intentions as
early coarse-grained communicative cues for the acquisition of
fine-grained meaning such as function words.

\begin{longtable}[]{@{}lll@{}}
\caption{\label{tab:speechActs} Definitions of the communicative functions
and their examples.}\tabularnewline
\toprule
\begin{minipage}[b]{0.14\columnwidth}\raggedright\strut
Function\strut
\end{minipage} & \begin{minipage}[b]{0.44\columnwidth}\raggedright\strut
Definitions\strut
\end{minipage} & \begin{minipage}[b]{0.33\columnwidth}\raggedright\strut
Examples\strut
\end{minipage}\tabularnewline
\midrule
\endfirsthead
\toprule
\begin{minipage}[b]{0.14\columnwidth}\raggedright\strut
Function\strut
\end{minipage} & \begin{minipage}[b]{0.44\columnwidth}\raggedright\strut
Definitions\strut
\end{minipage} & \begin{minipage}[b]{0.33\columnwidth}\raggedright\strut
Examples\strut
\end{minipage}\tabularnewline
\midrule
\endhead
\begin{minipage}[t]{0.14\columnwidth}\raggedright\strut
Descriptions\strut
\end{minipage} & \begin{minipage}[t]{0.44\columnwidth}\raggedright\strut
Describing what the world is like or asking about it. The primary goal
is to inform the addressee about how things are.\strut
\end{minipage} & \begin{minipage}[t]{0.33\columnwidth}\raggedright\strut
\enquote{\emph{It's not in the ditch or the drain pipe.}}\strut
\end{minipage}\tabularnewline
\begin{minipage}[t]{0.14\columnwidth}\raggedright\strut
Identifications\strut
\end{minipage} & \begin{minipage}[t]{0.44\columnwidth}\raggedright\strut
Identifying the category membership or an attribute of an object.
Speaker has uncertainty. A subtype of \enquote{Description}.\strut
\end{minipage} & \begin{minipage}[t]{0.33\columnwidth}\raggedright\strut
\enquote{\emph{Is that a ball or a balloon honey?}}\strut
\end{minipage}\tabularnewline
\begin{minipage}[t]{0.14\columnwidth}\raggedright\strut
Definitions and Examples\strut
\end{minipage} & \begin{minipage}[t]{0.44\columnwidth}\raggedright\strut
Providing labels for a category or examples for it. Speaker is certain.
Subtype of Description.\strut
\end{minipage} & \begin{minipage}[t]{0.33\columnwidth}\raggedright\strut
\emph{\enquote{This is a cup or a mug.} -- \enquote{berries like
blueberry or raspberry}}\strut
\end{minipage}\tabularnewline
\begin{minipage}[t]{0.14\columnwidth}\raggedright\strut
Preferences\strut
\end{minipage} & \begin{minipage}[t]{0.44\columnwidth}\raggedright\strut
Asking what the addressee wants or would like or stating what the
speaker wants or would like\strut
\end{minipage} & \begin{minipage}[t]{0.33\columnwidth}\raggedright\strut
\emph{\enquote{Do you wanna play pizza or read the book?}}\strut
\end{minipage}\tabularnewline
\begin{minipage}[t]{0.14\columnwidth}\raggedright\strut
Options\strut
\end{minipage} & \begin{minipage}[t]{0.44\columnwidth}\raggedright\strut
Either asking or listing what one can or is allowed to do. Giving
permission, asking for permission, or describing the possibilities.
Often the modal \enquote{can} is either present or can be
inserted.\strut
\end{minipage} & \begin{minipage}[t]{0.33\columnwidth}\raggedright\strut
\emph{\enquote{You could have wheat or rice.}}\strut
\end{minipage}\tabularnewline
\begin{minipage}[t]{0.14\columnwidth}\raggedright\strut
Directives\strut
\end{minipage} & \begin{minipage}[t]{0.44\columnwidth}\raggedright\strut
Directing the addressee to act or not act in a particular way. Common
patterns include \enquote{let's do \ldots{}}, \enquote{Why don't you do
\ldots{}}, or prohibitions such as \enquote{Don't \ldots{}}. The
difference with \enquote{options} is that the speaker expects the
directive to be carried out by the addressee. There is no such
expectation for \enquote{options}.\strut
\end{minipage} & \begin{minipage}[t]{0.33\columnwidth}\raggedright\strut
\emph{\enquote{let's go back and play with your ball or we'll read your
book.}}\strut
\end{minipage}\tabularnewline
\begin{minipage}[t]{0.14\columnwidth}\raggedright\strut
Clarifications\strut
\end{minipage} & \begin{minipage}[t]{0.44\columnwidth}\raggedright\strut
Something is said or done as a communicative act but the speaker has
uncertainty with respect to the form or the content.\strut
\end{minipage} & \begin{minipage}[t]{0.33\columnwidth}\raggedright\strut
\emph{\enquote{You mean boba or bubble?}}\strut
\end{minipage}\tabularnewline
\begin{minipage}[t]{0.14\columnwidth}\raggedright\strut
Repairs\strut
\end{minipage} & \begin{minipage}[t]{0.44\columnwidth}\raggedright\strut
Speaker correcting herself on something she said (self repair) or
correcting the addressee (other repair). The second disjunct is what
holds and is intended by the speaker. The speaker does not have
uncertainty with respect to what actually holds.\strut
\end{minipage} & \begin{minipage}[t]{0.33\columnwidth}\raggedright\strut
\emph{\enquote{There's an Oatio here, or actually, there's a wheat
here.}}\strut
\end{minipage}\tabularnewline
\begin{minipage}[t]{0.14\columnwidth}\raggedright\strut
Conditionals\strut
\end{minipage} & \begin{minipage}[t]{0.44\columnwidth}\raggedright\strut
Explaining in the second coordinand, what would follow if the first
coordinand is (or is not) followed. Subtype of Directive.\strut
\end{minipage} & \begin{minipage}[t]{0.33\columnwidth}\raggedright\strut
\emph{\enquote{Put that out of your mouth, or I'm gonna put it away.}}
-- \emph{\enquote{Come over here and I'll show you.}}\strut
\end{minipage}\tabularnewline
\begin{minipage}[t]{0.14\columnwidth}\raggedright\strut
Unconditionals\strut
\end{minipage} & \begin{minipage}[t]{0.44\columnwidth}\raggedright\strut
Denying the dependence of something on a set of conditions. Typical
format: \enquote{Whether X or Y, Z}. Subtype of Descriptions.\strut
\end{minipage} & \begin{minipage}[t]{0.33\columnwidth}\raggedright\strut
\emph{\enquote{Ready or not, here I come!}} (playing hide and
seek)\strut
\end{minipage}\tabularnewline
\bottomrule
\end{longtable}

\paragraph{Answer Type}\label{answer-type}

Whenever a parent's utterance was a polar question, the annotators coded
the utterance for the type of response it received from the children.
Table \ref{tab:answerTypes} shows the answer types in this study and
their definitions and examples. Utterances that were not polar questions
were simply coded as NA for this category. If children responded to
polar questions with \enquote{yes} or \enquote{no}, the category was YN
and if they repeated with one of the coordinands the category was AB. If
children said yes/no and followed it with one of the coordinands, the
answer type was determined as YN (yes/no). For example, if a child was
asked \enquote{Do you want orange juice or apple juice?} and the child
responded with \enquote{yes, apple juice}, our annotators coded the
response as YN. The reason is that in almost all cases, if a simple
yes/no response is felicitous, then it can also be optionally followed
with mentioning a disjunct. However, if yes/no is not a felicitous
response, then mentioning one of the alternatives is the only
appropriate answer. For example, if someone asks \enquote{Do you want to
stay here or go out?} a response such as \enquote{yes, go out} is
infelicitous and a better response is to simply say \enquote{go out}.
Therefore, we count responses with both yes/no and mentioning an
alternative as a yes/no response.

\begin{longtable}[]{@{}lll@{}}
\caption{\label{tab:answerTypes} Definitions of answer types and their
examples.}\tabularnewline
\toprule
\begin{minipage}[b]{0.18\columnwidth}\raggedright\strut
Type\strut
\end{minipage} & \begin{minipage}[b]{0.41\columnwidth}\raggedright\strut
Definitions\strut
\end{minipage} & \begin{minipage}[b]{0.33\columnwidth}\raggedright\strut
Examples\strut
\end{minipage}\tabularnewline
\midrule
\endfirsthead
\toprule
\begin{minipage}[b]{0.18\columnwidth}\raggedright\strut
Type\strut
\end{minipage} & \begin{minipage}[b]{0.41\columnwidth}\raggedright\strut
Definitions\strut
\end{minipage} & \begin{minipage}[b]{0.33\columnwidth}\raggedright\strut
Examples\strut
\end{minipage}\tabularnewline
\midrule
\endhead
\begin{minipage}[t]{0.18\columnwidth}\raggedright\strut
No Answer\strut
\end{minipage} & \begin{minipage}[t]{0.41\columnwidth}\raggedright\strut
The child provides no answer to the question.\strut
\end{minipage} & \begin{minipage}[t]{0.33\columnwidth}\raggedright\strut
Mother: \emph{\enquote{Would you like to eat some applesauce or some
carrots?}} Child: \emph{\enquote{Guess what Max!}}\strut
\end{minipage}\tabularnewline
\begin{minipage}[t]{0.18\columnwidth}\raggedright\strut
YN\strut
\end{minipage} & \begin{minipage}[t]{0.41\columnwidth}\raggedright\strut
The child responds with \emph{yes} or \emph{no}.\strut
\end{minipage} & \begin{minipage}[t]{0.33\columnwidth}\raggedright\strut
Father: \emph{\enquote{Can I finish eating one or two more bites of my
cereal?}} Child: \emph{\enquote{No.}}\strut
\end{minipage}\tabularnewline
\begin{minipage}[t]{0.18\columnwidth}\raggedright\strut
AB\strut
\end{minipage} & \begin{minipage}[t]{0.41\columnwidth}\raggedright\strut
The child responds with one of the disjuncts (alternatives).\strut
\end{minipage} & \begin{minipage}[t]{0.33\columnwidth}\raggedright\strut
Mother: \emph{\enquote{Is she a baby elephant or is she a toddler
elephant?}} Child: \emph{\enquote{It's a baby. She has a tail.}}\strut
\end{minipage}\tabularnewline
\bottomrule
\end{longtable}

\subsubsection{Inter-annotator
Reliability}\label{inter-annotator-reliability}

To train annotators and confirm their reliability for disjunction
examples, two annotators coded the same 240 instances of disjunction.
The inter-annotator reliability was calculated over 8 iterations of 30
examples each. After each iteration, annotators met to discuss
disagreements and resolve them. They also decided whether the category
definitions or annotation criteria needed to be made more precise.
Training was completed after three consecutive iterations showed
substantial agreement between the annotators for all categories (Cohen's
\(\kappa > 0.7\)). Figure \ref{fig:oReliabilityPlot} shows the
percentage agreement and the kappa values for each annotation category
over the 8 iterations.

\begin{figure}[tb]

{\centering \includegraphics{figs/oReliabilityPlot-1} 

}

\caption{Inter-annotator agreement for disjunction examples.}\label{fig:oReliabilityPlot}
\end{figure}

Agreement in the following three categories showed substantial
improvement after better and more precise definitions and annotation
criteria were developed: connective interpretation, intonation, and
communicative function. First, connective interpretation showed major
improvements after annotators developed more precise criteria for
selecting the propositions under discussion and separately wrote down
the two propositions connected by the connective word. For example, if
the original utterance was \enquote{do you want milk or juice?}, the
annotators wrote \enquote{you want milk, you want juice} as the two
propositions under discussion. This exercise clarified the exact
propositions under discussion and sharpened annotator intuitions with
respect to the connective interpretation that is communicated by the
utterance. Second, annotators improved agreement on intonation by
reconstructing an utterance's intonation for all three intonation
categories. For example, the annotator would examine the same sentence
\enquote{do you want coffee or tea?} with a rise-fall, a rise, and a
flat intonation. Then the annotator would listen to the actual utterance
and see which one most resembled the actual utterance. This method
helped annotators judge the intonation of an utterance more accurately.
Finally, agreement on communicative functions improved as the
definitions were made more precise. For example, the definition of
\enquote{directives} in Table \ref{tab:speechActs} explicitly mentions
the difference between \enquote{directives} and \enquote{options}.
Clarifying the definitions of communicative functions helped improve
annotator agreement.

Inter-annotator reliability for conjunction was calculated in the same
way. Two different annotators coded 300 utterances of \emph{and}.
Inter-annotator reliability was calculated over 10 iterations of 30
examples. Figure \ref{fig:andReliabilityPlot} shows the percentage
agreement between the annotators as well as the kappa values for each
iteration. Despite high percentage agreement between annotators, the
kappa values did not pass the set threshold of 0.7 in three consecutive
iterations. This paradoxical result is mainly due to a property of
kappa. An imbalance in the prevalence of annotation categories can
drastically lower its value. When one category is extremely common with
high agreement while other categories are rare, kappa will be low
(Cicchetti \& Feinstein, 1990; Feinstein \& Cicchetti, 1990). In almost
all annotated categories for conjunction, there was one class that was
extremely prevalent. In such cases, it is more informative to look at
the class specific agreement for the prevalent category than the overall
agreement measured by Kappa (Cicchetti \& Feinstein, 1990; Feinstein \&
Cicchetti, 1990).

\begin{figure}[tb]

{\centering \includegraphics{figs/andReliabilityPlot-1} 

}

\caption{Inter-annotator agreement for conjunction examples.}\label{fig:andReliabilityPlot}
\end{figure}

Table \ref{tab:andAgreeStats} lists the dominant classes as well as
their prevalence, the values of class specific agreement index, and
category agreement index (Kappa). Class specific agreement index is
defined as \(2n_{ii}/n_{i.}+n_{.i}\), where \(i\) represents the class's
row/column number in the category's confusion matrix, \(n\) the number
of annotations in a cell, and the dot ranges over all the row/column
numbers (Fleiss, Levin, \& Paik, 2013, p. 600; Ubersax, 2009). The class
specific agreement indices are high for all the most prevalent classes
showing that the annotators had very high agreement on these class, even
though the general agreement index (Kappa) was often low. The most
extreme case is the category \enquote{consistency} where almost all
instances were annotated as \enquote{consistent} with perfect class
specific agreement but low overall Kappa. In the case of utterance type
and syntactic level where the distribution of instances across classes
was more even, the general index of agreement Kappa is also high. In
general, examples of conjunction showed little variability across
annotation categories and mostly fell into one class within each
category. Annotators had high agreement for these dominant classes.

\begin{table}

\caption{\label{tab:andAgreeStats}Most prevalent annotation class in each annotation category with the values of class agreement indeces and category agreement indeces (Kappa).}
\centering
\begin{tabular}[t]{l|l|r|r|r}
\hline
Annotation Category & Class & Prevalence & Class Agreement Index & Kappa\\
\hline
intonation & flat & 0.86 & 0.89 & 0.24\\
\hline
interpretation & AND & 0.96 & 0.98 & 0.39\\
\hline
answer & NA & 0.84 & 0.94 & 0.67\\
\hline
utterance\_type & declarative & 0.76 & 0.94 & 0.70\\
\hline
communicative\_function & description & 0.77 & 0.90 & 0.59\\
\hline
syntactic\_level & clausal & 0.67 & 0.91 & 0.70\\
\hline
consistency & consistent & 0.99 & 1.00 & 0.50\\
\hline
\end{tabular}
\end{table}

\subsubsection{Results}\label{results}

First, I show the results for the study's dependent measure\footnote{All
  the confidence intervals shown in the plots for this section are
  simultaneous multinomial confidence intervals computed using the Sison
  and Glaz (1995) method.}. Figure \ref{fig:interpretationPlot} shows
the distribution of the connective interpretations in the study. The
most common interpretation was the conjunctive interpretation (AND,
49\%) followed by the exclusive interpretation (XOR, 35\%). Figure
\ref{fig:connectivePlot} shows the distribution of connective
interpretations by the connective words \emph{and} and \emph{or}. For
\emph{and}, the most frequent interpretation (in fact almost the only
interpretation), is conjunction AND. For \emph{or}, the most frequent
interpretation is exclusive disjunction XOR. These results replicate the
findings of Morris (2008).

\begin{figure}[tb]

{\centering \includegraphics{figs/interpretationPlot-1} 

}

\caption{The proportion of different interpretations of the connectives \textit{and/or} in child-directed speech}\label{fig:interpretationPlot}
\end{figure}

\begin{figure}[tb]

{\centering \includegraphics{figs/connectivePlot-1} 

}

\caption{Interpretations of \textit{and/or} in child-directed speech}\label{fig:connectivePlot}
\end{figure}

Based on these results, Morris argued that given the high frequency of
conjunction and exclusive disjunction in the input, children should map
the meanings of \emph{and} and \emph{or} as conjunction and exclusive
disjunction, at least initially, between the ages of 2 and 5 years.
According to Morris (2008), children learn the inclusive interpretation
of disjunction later as they encounter more inclusive (logical) uses of
\emph{or}. However, comprehension tasks show that children between 3 and
5 tend to interpret \emph{or} as inclusive disjunction rather than
exclusive disjunction in a variety of declarative sentences (Chierchia,
Crain, Guasti, Gualmini, \& Meroni, 2001; Gualmini, Crain, \& Meroni,
2000; Gualmini, Meroni, \& Crain, 2000, among others; Notley, Zhou,
Jensen, \& Crain, 2012). How can children learn the inclusive semantics
of \emph{or} if they rarely hear it? This is the puzzle of learning
disjunction, discussed in the first chapter. The remainder of this
section explores the role of cues that could help children successfully
interpret a disjunction as inclusive or exclusive.

\begin{figure}[tb]

{\centering \includegraphics{figs/utterancetypePlot-1} 

}

\caption{Connective interpretations in different sentence types.}\label{fig:utterancetypePlot}
\end{figure}

I will first look at the effect of utterance type on the interpretation
of \emph{or}. Figure \ref{fig:utterancetypePlot} shows the distribution
of connective interpretations in declarative, interrogative, and
imperative sentences. Interrogatives are more likely to be interpreted
as exclusive disjunction (XOR), imperatives are more likely to be
interpreted as inclusive (IOR) or exclusive (XOR), and declaratives are
most likely exclusive (XOR) or conjunctive (AND). It is important to
note here that the inclusive interpretations of imperatives are largely
due to invitations to action such as \enquote{Have some food or drink!}.
Such invitational imperatives seem to convey inclusivity (IOR)
systematically. They are often used to give the addressee full
permission with respect to both alternatives and it seems quite odd to
use them to imply exclusivity (e.g. \enquote{Have some food or drink but
not both!}), and they do not seem to be conjunctive either (e.g.
\enquote{Have some food and have some drink}). They rather imply that
the addressee is invited to have food, drink, or both.

\begin{figure}[tb]

{\centering \includegraphics{figs/intonationPlot-1} 

}

\caption{The distribution of connective interpretations in flat, rising, and rise-fall intonation.}\label{fig:intonationPlot}
\end{figure}

Figure \ref{fig:intonationPlot} shows the proportions of different
connective interpretations in the three intonation contours: flat, rise,
and rise-fall. A disjunction with a rise-fall intonation is most likely
interpreted as exclusive (XOR). If the intonation is rising, a
disjunction is more likely to be interpreted as inclusive (IOR). And a
disjunction with a flat intonation may be interpreted as exclusive
(XOR), conjunctive (AND), or inclusive (IOR). These results are
consistent with Pruitt and Roelofsen (2013)'s experimental findings that
a rise-fall intonation contour on a disjunction results in an exclusive
interpretation. Since rise-fall and rising intonation contours are
almost always on interrogatives, Figures \ref{fig:utterancetypePlot} and
\ref{fig:intonationPlot}, suggest that the rise-fall and rising
intonation types distinguish exclusive and inclusive interpretations of
disjunction in interrogatives. Furthermore, given a flat intonation
type, an imperative may be more likely to be inclusive (IOR).

\begin{figure}[tb]

{\centering \includegraphics{figs/consistencyPlot-1} 

}

\caption{Connective interpretations in disjunctions with consistent and inconsistent disjuncts.}\label{fig:consistencyPlot}
\end{figure}

Figure \ref{fig:consistencyPlot} shows the proportions of connective
interpretations in disjunctions with consistent vs.~inconsistent
disjuncts. When the disjuncts were consistent, the interpretation could
be exclusive (XOR), inclusive (IOR), or conjunctive (AND). When the
disjuncts were inconsistent, a disjunction almost always received an
exclusive interpretation. These results suggest that the exclusive
interpretation of a disjunction often stems from the inconsistent or
contradictory nature of the disjuncts themselves and not necessarily the
connective word \emph{or}. It should be noted here that in all
\emph{and}-examples, the disjuncts were consistent. This is not
surprising given that inconsistent meanings with \emph{and} result in a
contradiction. The only exception to this was one example where the
mother was mentioning two words as antonyms: \enquote{short and tall}.
This example is quite different from the normal utterances given that it
is meta-linguistic and list words rather than asserting the content of
the words.

\begin{figure}[tb]

{\centering \includegraphics{figs/consistencyByintonationPlot-1} 

}

\caption{Interpretations of and/or in the three intonation contours flat, rising, and rise-fall.}\label{fig:consistencyByintonationPlot}
\end{figure}

In Figure \ref{fig:consistencyByintonationPlot}, I break down connective
interpretations by both intonation and consistency. The results show
that disjunctions are interpreted as exclusive XOR when they carry
either inconsistent disjuncts or a rise-fall intonation. If the
disjunction has consistent disjuncts and carries a rising intonation, it
is most likely interpreted as inclusive IOR. Disjunctions with
consistent disjuncts and a flat intonation contour could have
conjunctive (AND), inclusive (IOR), or exclusive (XOR) interpretations.

\begin{figure}[tb]

{\centering \includegraphics{figs/syntaxPlot-1} 

}

\caption{Connective interpretations in clausal and sub-clausal disjunctions.}\label{fig:syntaxPlot}
\end{figure}

Figure \ref{fig:syntaxPlot} shows connective interpretations by the
syntactic level of the disjunction. As a reminder, we annotated
disjunctions with clausal/verbal disjuncts as \enquote{clausal}, and
those with other syntactic categories as sub-clausal. The goal was to
assess the role of syntax in the interpretation of disjunction. The
results suggest a small effect of clausal level disjuncts. Disjunctions
are more likely to be interpreted as exclusive when their disjuncts are
clauses or verbs rather than nominals, adjectives, or prepositions (all
sub-clausal units).

\begin{figure}[tb]

{\centering \includegraphics{figs/speechActPlot-1} 

}

\caption{Connective interpretations in different communicative functions.}\label{fig:speechActPlot}
\end{figure}

For the last independent variable in our study, I take a look at how
disjunction interpretations are affected by the communicative function
of the utterance they appear in. Figure \ref{fig:speechActPlot} shows
the proportions of connective interpretations in the 10 different
communicative functions of this study. The results show that certain
functions increase the likelihood of some connective interpretations. An
exclusive (XOR) interpretation of \emph{or} is common in acts of
clarification, identification, stating/asking preferences,
stating/asking about a description, or making a conditional statements.
These results are consistent with expectations on the communicative
intentions of that these utterances carry. In clarifications, the
speaker needs to know which of two alternatives the other party meant.
Similarly in identifications, speaker needs to know which category does
a referent belongs to. In preferences, parents seek to know which of two
alternatives the child wants. Even though descriptions could be either
inclusive or exclusive, in the current sample, most descriptions were
questions about the state of affairs and required the child to provide
one of the alternatives as the answer. In conditionals such as
\enquote{come here or you are grounded}, the point of the threat is that
only one disjunct can be true: either \enquote{you come and you are not
grounded} or \enquote{you don't come and you are grounded}. This is
similar to an exclusive interpretation of \emph{or}.

Repairs often received an exclusive (XOR) or a second-disjunct-true
(NAB) interpretation. This is expected given that in repairs the speaker
intends to say that the first disjunct is incorrect or inaccurate.
Unconditionals and definitions/examples always had a conjunctive (AND)
interpretation. Again, this is to be expected. In such cases the speaker
intends to communicate that all options apply. If the mother says that
\enquote{cats are animals like lions or tigers}, she intends to say that
both lions and tigers are cats and not one or the other. Interestingly,
in some cases (not all), \emph{or} is replaceable by \emph{and}:
\enquote{cats are animals like lions and tigers}. In unconditionals, the
speaker communicates that in both alternatives, a certain proposition
holds. For example, if the mother says \enquote{ready or not, here I
come!}, she communicates that \enquote{I come} is true in both cases
where \enquote{you are ready} and \enquote{you are not ready}.

Options were often interpreted either as conjunctive (AND) or inclusive
(IOR). The category \enquote{options} contained examples of free-choice
inferences such as \enquote{you could drink orange juice or apple
juice}. This study found free-choice examples much more common than the
current literature on the acquisition of disjunction suggests. Finally,
directives received either an IOR or XOR interpretation. It is important
to note here that the most common communicative function in the data
were preferences and descriptions. Other communicative functions such as
unconditionals or options were fairly rare. Despite their infrequent
appearance, these constructions must be learned by children at some
point, since almost all adults know how to interpret them. It is clear
from the investigation here that any learning account for function word
meaning/interpretation also needs to account for how such infrequent
constructions are learned.

Finally, I take a look at how children responded to the questions
containing \emph{or}. As a reminder, we annotated every polar question
such as \enquote{do you want cereal or toast?} for the type of answer
children provided. An answer such as \enquote{yes/no} is annotated as YN
and an answer with alternatives such as \enquote{cereal/toast} is
annotated as \enquote{AB}. Figure \ref{fig:answerPlot} shows the monthly
proportions of these answer types between 1 and 3 years of age.
Initially, children provided no answer to polar questions, but by the
age of 3 years, the majority of such questions received a yes/no (YN) or
alternative (AB) answer.

\begin{figure}[tb]

{\centering \includegraphics{figs/answerPlot-1} 

}

\caption{The proportions of children's answer types to polar questions containing the connective \textit{or} at different ages (in months).}\label{fig:answerPlot}
\end{figure}

These two answer types are not appropriate for all types of polar
questions that contain \emph{or}. For example, alternative answers are
typically provided to alternative questions with the rise-fall
intonation. For example, a question such as \enquote{do you want to stay
here or go out?} receives an answer such as \enquote{stay-here/go-out}
and not \enquote{yes/no}. However, a polar disjunctive question such as
\enquote{do you want any tea or coffee?} typically receives a
\enquote{yes}/\enquote{no} rather than only one of the alternatives like
\enquote{tea/coffee}, even though both answers are possible.

Based on such typical responses patterns, we can define appropriate
answers to questions with disjunction in the following way: an
alternative (AB) answer is appropriate for an alternative questions
(with \enquote{or} and rise-fall intonation) and a \enquote{yes/no}
answer (YN) is appropriate for a polar question. Of course this
classification is too strict and misses some nuanced cases but it
provides a rough estimate of appropriate answers offered to parents'
questions. Figure \ref{fig:answerHitsPlot} shows the monthly proportion
of children's appropriate answers between the ages of 1 and 3. The
results show that even with a strict measure, children show an increase
in the proportion of their appropriate responses to questions containing
\emph{or} between 20 to 30 months of age (roughly 2 and 3 years of age).
This increase in appropriate responses is consistent with the results
from comprehension studies that suggest children's understanding of
\emph{and} and \emph{or} develops between 2 and 4 years of age.

\begin{figure}[tb]

{\centering \includegraphics{figs/answerHitsPlot-1} 

}

\caption{Proportion of children's appropriate resonses}\label{fig:answerHitsPlot}
\end{figure}

\subsubsection{Discussion}\label{discussion}

The goal of this study was to discover the potential cues in
child-directed speech that could help children learn the interpretations
of \emph{and} and \emph{or}. The study presented 1000 examples of
\emph{and} and \emph{or} in child-directed speech, annotated for their
truth-conditional interpretation, as well as five candidate cues to
their interpretation: (1) Utterance Type; (2) Intonation Type; (3)
Syntactic Level; (4) Conceptual Consistency; (5) Communicative Function.
Like Morris (2008), this study found that the most common
interpretations of \emph{and} and \emph{or} are conjunction AND and
exclusive disjunction XOR. When the data were broken down by the
connectives, \emph{and} was almost always interpreted as a conjunction
while \emph{or} received three main interpretations: exclusive
disjunction XOR, inclusive disjunction IOR, and conjunction AND.

While the most frequent interpretation of \emph{or} was exclusive XOR
overall followed by IOR, the distribution of disjunction interpretations
shifted when they were broken down by the cues identified here. A
disjunction was most likely exclusive if the alternatives were
inconsistent (i.e.~contradictory). A disjunction was most likely
exclusive if it appeared in a question. Within questions, a disjunction
was most likely exclusive if the intonation was rise-fall. If the
intonation was rising, the question was interpreted as inclusive. The
syntactic category of the disjuncts could also provide information for
interpretation. If the disjuncts were clausal then it was more likely
for the disjunction to be interpreted as exclusive, even though this
effect was small. Finally, specific communicative functions required
specific interpretations of the connective. \emph{Or} often received a
conjunctive interpretation in the following contexts: defining terms and
providing examples, enumerating options, and in unconditional
constructions. These results suggest that in order to successfully learn
to interpret a disjunction, children need to pay attention to a wide
variety of formal and conceptual factors.

In order to have a rough measure of children's comprehension of
disjunction, this study also investigated the types of answers they
provided to polar questions with disjunction. Between the ages of 20 and
30 months (roughly 1;6 to 2;6 years), children start to answer \emph{or}
questions appropriately. They would respond to a yes/no question such as
\enquote{do you want any apple juice or orange juice?} with a yes/no
answer. They would also respond to an alternative question, as
in\enquote{do you want to play inside or outside?}, with one of the
alternates, e.g. \enquote{inside}. This finding is consistent with the
first corpus study presented in this chapter, which reported that the
age range between 1;6 and 4 is the age range in which children develop
their understanding of \emph{and} and \emph{or}.

Due to the exploratory nature of this study, it is important to
replicate and extend these results and conclusions in future studies.
For example, future studies could use an automated procedure for the
annotation of categories such as utterance type, syntactic level, and
intonation. An automated procedure would also allow for the annotation
of larger samples and so could result in more reliable estimates for the
role of various factors in learning the meanings of function words. For
categories such as communicative function and connective interpretation,
future studies could use a larger number of independent annotators to
increase the speed and number of annotations. However, several results
reported in this study are independently supported by previous research.
Morris (2008) found similar results with respect to the overall
interpretation of disjunction in child-directed speech: \emph{and} is
most often interpreted as conjunction and \emph{or} as exclusive
disjunction. In an experimental study, Pruitt and Roelofsen (2013) have
shown that a rise-fall intonation results in an exclusive
interpretation. Geurts (2006) has argued that a portion of exclusivity
inferences are simply due to the fact that the alternatives are mutually
exclusive and inconsistent.

Finally, the list of cues investigated here is in no way exhaustive.
There are at least two additional, possibly important factors/cues that
I set aside due to the difficulties that their annotation would have
introduced. First, an exclusive interpretation is sometimes the result
of a presupposition that only one alternative can hold or would matter
for the purposes of the conversation. For example, in the context of a
class activity where students pair up, a statement such as \enquote{Lisa
worked with Ann or John} is interpreted as exclusive simply because the
context already presupposes that only one disjunct can be true. Second,
some exclusivity inferences are due to the speaker's choice of
connective, namely using \emph{or} rather than \emph{and}. Grice (1989)
famously argued that in some cases, we interpret a disjunction like
\emph{A or B} as A or B, but not both because we reason that if the
speaker intended to communicate that both alternatives hold, s/he would
have said \emph{A and B}. This study did not annotate for such cases.
However, the study's results suggest that such cases of exclusive
interpretations are less frequent in child-directed speech that the ones
already annotated for. Investigating how often such cases of pragmatic
exclusion appear in child-directed speech can help us better understand
the role of input in children's acquisition of scalar implicatures.

\section{Learning to interpret a
disjunction}\label{learning-to-interpret-a-disjunction}

Given the wide range of interpretations that \emph{or} can have, how can
children learn to interpret it correctly? This is what study 3
addresses. In doing so, it also provides a solution to the puzzle of
learning disjunction. To remind you about the puzzle, previous research
have shown that the majority of \emph{or}-examples children hear are
exclusive. However, comprehension studies report that between the ages
of three and five, children can interpret \emph{or} as inclusive
disjunction in declarative sentences (Crain, 2012). The finding of the
comprehension studies and the corpus studies taken together present a
learning puzzle: how can children learn to interpret \emph{or} as
inclusive if they mostly hear exclusive examples? This chapter provides
a solution by developing a cue-based account for children's acquisition
of connectives. More generally, the account proposed in this chapter is
helpful for learning words with multiple interpretations when one
interpretation dominates the learner's input. \#\# Cues to coordinator
meanings

Three important compositional cues can help learners in restricting
their hypotheses to coordinator meanings. First, as pointed out by
Haspelmath (2007), coordination has specific compositional properties.
Coordinators combine two or more units of the same type and return a
larger unit of the same type. The larger unit has the same semantic
relation with the surrounding words as the smaller units would have had
without coordination. These properties separate coordinators from other
function words such as articles, quantifiers, numerals, prepositions,
and auxiliaries which are not used to connect sentences or any two
similar units for that matter. In fact, the special syntactic properties
of coordinators have compelled syntactic theories to consider specific
rules for coordination.

The literature on syntactic bootstrapping suggests that children can use
syntactic properties of the input to limit their word meaning hypotheses
to the relevant domain (Brown, 1957; see Fisher, Gertner, Scott, \&
Yuan, 2010 for a review; Gleitman, 1990). In the current 1073
annotations of conjunction and disjunction, I found that \emph{and} and
\emph{or} connected sentences/clauses 56\% of the time. This pattern is
unexpected for any other class of function words and it is possible that
the syntactic distribution of coordinators cue the learners to the space
of sentential connective meanings.

Second, in the annotation study I found that \emph{and} never occurs
with inconsistent coordinands (e.g. \enquote{clean and dirty}) while
\emph{or} commonly does (e.g. \enquote{clean or dirty}). The
inconsistency of the coordinands can cue the learner to not consider
conjunction as a meaning for the coordinator given that a conjunctive
meaning would too often lead to a contradiction at the utterance level.
On the other hand, choosing disjunction as the meaning avoids this
problem. Third, the large scale study of Chapter \ref{corpus} found that
\emph{or} is more likely to occur in questions than statements while
\emph{and} is more likely in statements. Since questions often contain
more uncertainty while statements are more informative, it is possible
that these environments bias the learner towards selecting hypotheses
that match this general communicative function. Disjunction is less
informative than conjunction and it is possible that the frequent
appearance of \emph{or} in questions cues learners to both its meaning
as a disjunction as well as the ignorance inference commonly associated
with it.

Finally, it is reasonable to assume that not all binary connective
meanings shown in Figure \ref{fig:binaryLogicalConnectivess} are as
likely for mapping. For example, coordinators that communicate
tautologies or contradictions seem to be not good candidates for
informative communication. Similarly, if A coordinated with B simply
asserts the truth of A and says nothing about B, it is unclear why it
would be needed if the language already has the means of simply
asserting A. It is possible that pragmatic principles already bias the
hypothesis space to favor candidates that are communicatively more
efficient.

\begin{figure}[tb]

{\centering \includegraphics{figs/binaryLogicalConnectivess-1} 

}

\caption{The truth table for the 16 binary logical connectives. The rows represent the set of situations where zero, one, or both propositions are true. The columns represent the 16 possible connectives and their truth conditions. Green cells represent true situations.}\label{fig:binaryLogicalConnectivess}
\end{figure}

Even though these findings are suggestive, they need to be backed up by
further observational and experimental evidence to show that children do
actually use these cues in learning connective meanings. In the next
section, I turn to the more specific issue of learning the correct
interpretation of \emph{and} and \emph{or} from the input data. As in
the case of number words, previous research has provided insight into
how children comprehend a disjunction and what they hear from their
parents. The main question is how children learn what they comprehend
from what they hear. I turn to this issue in the next section.

\subsection{\texorpdfstring{Learning to interpret \emph{and} and
\emph{or}: A cue-based
account}{Learning to interpret and and or: A cue-based account}}\label{myaccount}

Previous comprehension studies have shown that children as early as age
three can interpret a disjunction as inclusive (see Crain, 2012 for an
overview). However, Morris (2008) showed that exclusive interpretations
are much more common than other interpretations of disjunction in
children's input. In Figure \ref{fig:interpretation}, I show the results
of Chapter \ref{corpus}\enquote{s annotation study by grouping the
disjunction interpretations into exclusive (EX) and inclusive (IN),
i.e.~non-exclusive categories. These results replicate Morris} (2008)
finding and reinforce a puzzle raised by Crain (2012): How can children
learn the inclusive interpretation of disjunction when the majority of
the examples they hear are exclusive? To answer this question, I draw on
insights from the Gricean approach to semantics and pragmatics discussed
in Chapter \ref{sempragLit}.

\begin{figure}[tb]

{\centering \includegraphics{figs/interpretation-1} 

}

\caption{Proportion of exclusive and inclusive interpretations of disjunction in child-directed speech. Error bars represent bootstrapped 95\% confidence intervals.}\label{fig:interpretation}
\end{figure}

Research in Gricean semantics and pragmatics has shown that the word
\emph{or} is not the only factor relevant to the interpretation of a
disjunction. It is not only the presence of the word \emph{or} that
leads us to interpret a disjunction as inclusive, exclusive, or
conjunctive, but rather the presence of \emph{or} along with several
other factors such as intonation (Pruitt \& Roelofsen, 2013), the
meaning of the disjuncts (Geurts, 2006), and the conversational
principles governing communication (Grice, 1989). The interpretation and
acquisition of the word \emph{or} cannot, therefore, be separated from
all the factors that accompany it and shape its final interpretation.

In the literature on word learning and semantic acquisition,
form-meaning mapping is often construed as mapping an isolated form such
as \emph{gavagai} to an isolated concept such as \enquote{rabbit}. While
this approach may be feasible for content words, it will not work for
function words such as \emph{or}. First, the word \emph{or} cannot be
mapped in isolation from its formal context. As Pruitt and Roelofsen
(2013) showed, the intonation that accompanies a disjunction affects its
interpretation. Therefore, a learner needs to pay attention to the word
\emph{or} as well as the intonation contour that accompanies it. Second,
the word \emph{or} cannot be mapped to its meaning isolated from the
semantics of the disjuncts that accompany it. As Geurts (2006) argued,
the exclusive interpretation is often enforced simply because the
options are incompatible. For example, \enquote{to be or not to be} is
exclusive simply because one cannot both be and not be. In addition,
conversational factors play an important role in the interpretation of
\emph{or} as Grice (1989) argued. In sum, the interpretation and
acquisition of function words such as \emph{or} require the learner to
consider the linguistic and nonlinguistic context of the word and map
the meanings accordingly.

Previous accounts have adopted a model in which a function word such as
\emph{or} is mapped directly to its most likely interpretation:

\emph{or} \(\rightarrow \oplus\)

This model is often used in cross-situational accounts of content words.
Here I argue that the direct mapping of \emph{or} to its interpretation
without consideration of its linguistic context is the primary cause of
the learning puzzle for \emph{or}. Instead, I propose that the word
\emph{or} is mapped to an interpretation in a context-dependent manner,
along with the interpretive cues that accompany it such as intonation
and disjunct semantics:

{[}connective: \emph{or}, Intonation: rise-fall, Disjuncts:
inconsistent{]} \(\rightarrow \oplus\)

{[}connective: \emph{or}, Intonation: rising, Disjuncts: consistent{]}
\(\rightarrow \lor\)

Figure \ref{fig:interpretationByIntonationAndConsistency} shows that the
rate of exclusive interpretations change systematically when the data
are broken down by intonation and consistency. Given a rise-fall
intonation contour, a disjunction is almost always interpreted as
exclusive. Similarly, if the propositions are inconsistent, the
disjunction is most likely interpreted as exclusive. When either of
these two features are absent, a disjunction is more likely to receive
an inclusive interpretation.

\begin{figure}[tb]

{\centering \includegraphics{figs/interpretationByIntonationAndConsistency-1} 

}

\caption{Exclusive and inclusive interpretations broken down by intonation (flat, rise, rise-fall) and consistency. Error bars represent bootstrapped 95\% confidence intervals.}\label{fig:interpretationByIntonationAndConsistency}
\end{figure}

In this account, it is not a single word that gets mapped to an
interpretation but rather a cluster of features. This method has two
advantages. First, it deals with the context dependency of disjunction
interpretation. The learner knows that \emph{or} with some intonation
has to be interpreted differently from one with another. Second, it
allows the learner to pull apart the contribution of \emph{or} from the
interpretive cues that often accompany it. In fact, analysis of all
mapping clusters in which \emph{or} participates and generalization over
them can help the learner extract the semantics of \emph{or} the way it
is intended by Gricean accounts of semantics/pragmatics. For those
skeptical of such an underlying semantics for \emph{or}, there is no
need for further analysis of the mapping clusters. The meaning of
\emph{or} as a single lexical item is distributed among the many
mappings in which it participates. In the next section, I implement this
idea using decision tree learning.

\subsection{Modeling Using Decision Tree Learning}\label{DecisionTrees}

A decision tree is a classification model structured as a hierarchical
tree with nodes, branches, and leaves (Breiman, 2017). The tree starts
with an initial node, called the root, and branches into more nodes
until it reaches the leaves. Each node represents the test on a feature,
each branch represents an outcome of the test, and each leaf represents
a classification label. Using a decision tree, observations can be
classified or labeled based on a set of features.

\emph{I personally wouldn't include this example in a paper,
unnecessary?} For example, we can make a decision tree to predict
whether a food item is a fruit or not based on its color (green or not)
and shape (round or not). An example decision tree is the following: at
the root, the model can ask whether the item is green or not. If yes,
the model creates a leaf and labels the item as \enquote{not fruit}. If
not, the model creates another node and asks if the item is round. If
yes, the item is classified as a \enquote{fruit} and if not it is
classified as \enquote{not fruit}.

Decision trees have several advantages for modeling cue-based accounts
of semantic acquisition. First, decision trees use a set of features to
predict the classification of observations. This is analogous to using
cues to predict the correct interpretation of a word or an utterance.
Second, unlike many other machine learning techniques, decision trees
result in models that are interpretable. Third, the order of decisions
or features used for classification is determined based on information
gain. Features that appear higher (earlier) in the tree are more
informative and helpful for classification. Therefore, decision trees
can help us understand which cues are probably more helpful for the
acquisition and interpretation of a word.

Decision tree learning is the construction of a decision tree from
labeled training data. This section applies decision tree learning to
the annotated data of Chapter \ref{corpus} by constructing random
forests (Breiman, 2001; Ho, 1995). In random forest classification,
multiple decision trees are constructed on subsets of the data, and each
tree predicts a classification. The ultimate outcome is a majority vote
of each trees classification. Since decision trees tend to overfit data,
random forests control for overfitting by building more trees and
averaging their results. \textbf{(Citation)} Next section discusses the
methods used in constrcting the random forests for interpreting
connectives \emph{or}/\emph{and}.

\subsubsection{Methods}\label{methods-2}

The random forest models were constructed using python's Sci-kit Learn
package (Pedregosa et al., 2011). The annotated data had a feature array
and a connective interpretation label for each connective use.
Connective interpretations included exclusive (XOR), inclusive (IOR),
conjunctive (AND), negative inclusive (NOR), and NPQ which states that
only the second proposition is true. The features or cues used included
all other annotation categories: intonation, consistency, syntactic
level, utterance type, and communicative function. All models were
trained with stratified 10-Fold cross-validation to reduce overfitting.
Stratified cross-validation maintains the distribution of the initial
data in the random sampling to build cross validated models. Maintaining
the data distribution ensures a more realistic learning environment for
the forests. Tree success was measured with F1-Score, harmonic average
of precision and recall \textbf{(Citation)}.

First a grid search was run on the hyperparamter space to establish the
number of trees in each forest and the maximum tree depth allowable. The
grid search creates a grid of all combinations of forest size and tree
depth and then trains each forest from this grid on the data. The
forests with the best F1-score and lowest size/depth are reported.
**(Citation*)\textbf{ The default number of trees for the forests was
set to 20, with a max depth of eight and a minimum impurity decrease of
0. Impurity was measured with gini impurity, which states the odds that
a random member of the subset would be mislabled if it were randomly
labeled according to the distribution of labels in the subset.
}(Citation)**

Decision trees were fit with high and low minimum gini decrease values.
High minimum gini decrease results in a tree that does not use any
features for branching. Such a tree represents the baseline or
traditional approach to mapping that directly maps a word to its most
likely interpretation. Low minimum gini decrease allows for a less
conservative tree that uses multiple cues/features to predict the
interpretation of a disjunction. Such a tree represents the cue-based
context-sensitive account of word learning discussed in the previous
section.

\subsubsection{Results}\label{results-1}

We first present the results of the random forests in the binary
classification task. The models were trained to classify exclusive and
inclusive interpretations of disjunction. For visualization of trees, we
selected the highest performing tree in the forest by testing each tree
and selecting for highest F1 score. While the forests performance is not
identical to the highest performing tree, the best tree gives an
illustrative example of how the tree performs.

Figure \ref{fig:binaryBaseline} shows the best performing decision tree
with high minimum gini decrease. As expected, a learner that does not
use any cues would interpret \emph{or} as exclusive all the time. This
is the baseline model. Figure \ref{fig:binaryCueBased} shows the best
performing decision tree with low minimum gini decrease. The tree has
learned to use intonation and consistency to classify disjunctions as
exclusive or inclusive. As expected, if the intonation is rise-fall or
the disjuncts are inconsistent, the interpretation is exclusive.
Otherwise, the disjunction is classified as inclusive.

\begin{figure}
\centering
\includegraphics{figs/binaryBaseline-1.pdf}
\caption{\label{fig:binaryBaseline}Baseline tree grown with minimum impurity
decrease of 0.2. The tree always classifies examples of disjunction as
exclusive.}
\end{figure}

\begin{figure}
\centering
\includegraphics{figs/binaryCueBased-1.pdf}
\caption{\label{fig:binaryCueBased}Cue-based tree grown with minimum
impurity decrease of 0.01. The tree classifies examples of disjunction
with rise-fall intonation as exclusive (intonation \textgreater{} 1.5).
If the intonation is not rise-fall but the disjuncts are inconsistent
(consistency \textless{} 0.5), then the disjunction is still classified
as exclusive. However, if neither of these two hold, the disjunction is
classified as inclusive.}
\end{figure}

Figure \ref{fig:XorBinary} shows the average F1 scores of the baseline
and cue-based models in classifying exclusive examples. The models
perform relatively well and similar to each other, but the cue-based
model performs slightly better. The real difference between the baseline
model and the cue-based model is in their performance on inclusive
examples. Figure \ref{fig:IorBinary} shows the F1 score of the forests
as a function of the training size in classifying inclusive examples. As
expected, the baseline model performs very poorly while the cue-based
model does a relatively good job and improves with more examples.

\begin{figure}
\centering
\includegraphics{figs/XorBinary-1.pdf}
\caption{\label{fig:XorBinary}The average F1 score for class XOR (exclusive)
as a function of the number of training examples in the baseline and
cue-based models. The colored shades show the 95\% confidence
intervals.}
\end{figure}

\begin{figure}
\centering
\includegraphics{figs/IorBinary-1.pdf}
\caption{\label{fig:IorBinary}The average F1 score for class IOR (inclusive)
as a function of the number of training examples in the baseline and
cue-based models. The colored shades show the 95\% confidence
intervals.}
\end{figure}

Next, we use decision tree learning in a ternary classification task.
The model uses features to interpret a coordination with \emph{and} and
\emph{or} as inclusive (IOR), exclusive (XOR), or conjunctive (AND).
Figure \ref{fig:ternaryBaseline} shows the baseline decision tree with
high minimum gini decrease, which only uses the presence of the words
\emph{or}/\emph{and} to interpret conjunction and disjunction. As
expected, the tree interprets a coordination with \emph{and} as a
conjunction and one with \emph{or} as exclusive disjunction. Figure
\ref{fig:ternaryCueBased} shows the cue-based decision tree with low
minimum gini decrease. In addition to the presence of \emph{and} and
\emph{or}, the tree uses intonation, consistency, communicative
function, and utterance type to distinguish exclusive, inclusive, and
conjunctive uses of disjunction. In short, a disjunction that is
rise-fall, inconsistent, or has a conditional communicative function is
classified as exclusive. Otherwise the disjunction is classified as
inclusive. The tree also finds conjunctive interpretations of
disjunction more likely in declarative sentences than interrogatives.

\begin{figure}
\centering
\includegraphics{figs/ternaryBaseline-1.pdf}
\caption{\label{fig:ternaryBaseline}The baseline tree grown on conjunctions
and disjunctions with minimum impurity decrease of 0.2. The tree uses
the words \textit{and/or} and classifies them as conjunction and
exclusive disjunction respectively.}
\end{figure}

\begin{figure}
\centering
\includegraphics{figs/ternaryCueBased-1.pdf}
\caption{\label{fig:ternaryCueBased}The cue-based tree grown on conjunctions
and disjunctions with minimum impurity decrease of 0.01. After using the
words \textit{and/or}, the tree uses intonation, consistency, and the
conditional communicative function to classify a large number of
exclusive cases. Then it uses utterance type (interrogative) to label
inclusive cases.}
\end{figure}

Figure \ref{fig:ANDintermediate} shows the average F1 score of the
conjunctive interpretations (AND) for the baseline and the cue-based
models. Since the vast majority of the conjunctive interpretations are
predicted by the presence of the word \emph{and}, the baseline and
cue-based models show similar performances. Setting aside conjunction
examples, Figure \ref{fig:ANDintermediateDis} shows the average F1 score
of the AND interpretation of disjunction only. Here we see that the
cue-based model performs better than the default model in guessing
conjunctive interpretations of disjunction. The informal analysis of the
trees suggest that the model does this by using the \enquote{speech act}
cue. Figure \ref{fig:XORintermediate} shows the average F1-score of the
exclusive interpretations (XOR) for the baseline and the cue-based
models. The cue-based model does slightly better than the baseline
model. As before, the most important improvement comes in identifying
inclusive examples. Figure \ref{fig:IORintermediate} shows the average
F1-score of the inclusive interpretations (IOR) for both baseline and
cue-based models. The baseline model performs very poorly while the
cue-based model is capable of classifying inclusive examples as well.

\begin{figure}
\centering
\includegraphics{figs/ANDintermediate-1.pdf}
\caption{\label{fig:ANDintermediate}The average F1 score for class AND as a
function of the number of training examples in the baseline and
cue-based models. The colored shades show the 95\% confidence
intervals.}
\end{figure}

\begin{figure}
\centering
\includegraphics{figs/ANDintermediateDis-1.pdf}
\caption{\label{fig:ANDintermediateDis}The average F1 score for class AND of
disjunction examles as a function of the number of training examples in
the baseline and cue-based models. The colored shades show the 95\%
confidence intervals.}
\end{figure}

\begin{figure}
\centering
\includegraphics{figs/XORintermediate-1.pdf}
\caption{\label{fig:XORintermediate}The average F1 score for class XOR as a
function of the number of training examples in the baseline and
cue-based models. The colored shades show the 95\% confidence
intervals.}
\end{figure}

\begin{figure}
\centering
\includegraphics{figs/IORintermediate-1.pdf}
\caption{\label{fig:IORintermediate}The average F1 score for class IOR as a
function of the number of training examples in the baseline and
cue-based models. The colored shades show the 95\% confidence
intervals.}
\end{figure}

Finally, welook at decision trees trained on the annotation data to
predict all the interpretation classes for disjunction: AND, XOR, IOR,
NOR, and NPQ. Figure \ref{fig:wholeBaseline} shows the baseline model
that only uses the words \emph{and} and \emph{or} to classify. As
expected, \emph{and} receives a conjunctive interpretation (AND) and
\emph{or} receives an exclusive interpretation (XOR). Figure
\ref{fig:wholeCueBased} shows the best example tree of the cue-based
model. The leaves of the tree show that it recognizes exclusive,
inclusive, conjunctive, and even negative inclusive (NOR)
interpretations of disjunction. How does the tree achieve that? Like the
baseline model, the tree first asks about the connective used:
\emph{and} vs. \emph{or}. Then like the previous models, it asks about
intonation and consistency. If the intonation is rise-fall, or the
disjuncts are inconsistent, the interpretation is exclusive. Then it
asks whether the sentence is an interrogative or a declarative. If
interrogative, it guesses an inclusive interpretation. This basically
covers questions with a rising intonation. Then the tree picks
declarative examples that have conditional speech act (e.g.
\enquote{give me the toy or you're grounded}) and labels them as
exclusive. Finally, if negation is present in the sentence, the tree
labels the disjunction as NOR.

\begin{figure}
\centering
\includegraphics{figs/wholeBaseline-1.pdf}
\caption{\label{fig:wholeBaseline}The baseline tree grown on conjunctions
and disjunctions with minimum impurity decrease of 0.2. The tree uses
the words \textit{and/or} and classifies them as conjunction and
exclusive disjunction.}
\end{figure}

\begin{figure}
\centering
\includegraphics{figs/wholeCueBased-1.pdf}
\caption{\label{fig:wholeCueBased}The cue-based tree grown on conjunctions
and disjunctions with minimum impurity decrease of 0.01. After using the
words \textit{and/or}, the tree uses intonation and consistency to
classify a large number of exclusive cases. Then it uses utterance type
(interrogative) to label many inclusive cases, as well as the
communicative function (conditional) to catch more exclusive examples.
Finally, it asks whether the sentence has negation or not. If so, it
classifies the negative inlusive examples as NOR.}
\end{figure}

Figures \ref{fig:ANDWhole}, \ref{fig:XORWhole}, and \ref{fig:IORWhole}
show the average F1-scores for the conjunctive (AND), exclusive (XOR),
and inclusive (IOR) interpretations as a function of training size. The
results are similar to what wereported before with the ternary
classification. While the cue-based model generally performs better than
the baseline model, it shows substantial improvement in classifying
inclusive cases.

\begin{figure}
\centering
\includegraphics{figs/ANDWhole-1.pdf}
\caption{\label{fig:ANDWhole}The average F1 score for class AND as a
function of the number of training examples in the baseline and
cue-based models. The colored shades show the 95\% confidence
intervals.}
\end{figure}

\begin{figure}
\centering
\includegraphics{figs/XORWhole-1.pdf}
\caption{\label{fig:XORWhole}The average F1 score for class XOR as a
function of the number of training examples in the baseline and
cue-based models. The colored shades show the 95\% confidence
intervals.}
\end{figure}

\begin{figure}
\centering
\includegraphics{figs/IORWhole-1.pdf}
\caption{\label{fig:IORWhole}The average F1 score for class IOR as a
function of the number of training examples in the baseline and
cue-based models. The colored shades show the 95\% confidence
intervals.}
\end{figure}

Figure \ref{fig:NORWhole} shows the average F1-score for the negative
inclusive interpretation as a function of training size. Compared to the
baseline model, the cue-based model shows a substantially better
performance in classifying negative sentences. The success of the model
in classifying negative inclusive examples (NOR) suggests that the
cue-based model offers a promising approach for capturing the scope
relation of operators such as negation and disjunction. Here, the model
learns that when negation and disjunction are present, the sentence
receives a negative inclusive (NOR) interpretation. In other words, the
model has learned the narrow-scope interpretation of negation and
disjunction from the input data. In a language where negation and
disjunction receive an XOR interpretation (not A or not B), the
cue-based model can learn the wide-scope interpretation of disjunction.

\begin{figure}
\centering
\includegraphics{figs/NORWhole-1.pdf}
\caption{\label{fig:NORWhole}The average F1 score for class NOR as a
function of the number of training examples in the baseline and
cue-based models. The colored shades show the 95\% confidence
intervals.}
\end{figure}

Finally, Figure \ref{fig:NPQWhole} shows the average F1 score for the
class NPQ. This interpretation suggested that the first disjunct is
false but the second true. It was seen in examples of repair most often
and the most likely cue to it was also the communicative function or
speech act of repair. The results show that even though there were
improvements in the cue-based model, they were not stable as shown by
the large confidence intervals. It is possible that with larger training
samples, the cue-based model can reliably classify the NPQ
interpretations as well.

\begin{figure}
\centering
\includegraphics{figs/NPQWhole-1.pdf}
\caption{\label{fig:NPQWhole}The average F1 score for class NPQ as a
function of the number of training examples in the baseline and
cue-based models. The colored shades show the 95\% confidence
intervals.}
\end{figure}

\subsection{Discussion}\label{discussion-1}

In this chapter, we discussed two accounts for the acquisition of
function words. The first account was a baseline (context-independent)
account that is used in vanilla cross-situational word learning: words
are isolated and directly mapped to their most frequent meanings. The
second account is what I called the cue-based context-dependent mapping
in which words are mapped to meanings conditional on a set of present
cues in the context. I argued that the puzzle of learning disjunction
arises because in the baseline account, forms are mapped directly to
meanings without considering the context of use. Under this account, the
input statistics supports an exclusive interpretation for \emph{or}.
However, comprehension studies show that children can interpret
\emph{or} as inclusive. I showed that the cue-based account resolves
this problem by allowing \emph{or} to be mapped to its interpretation
according to the set of contextual cues that disambiguate it. The
results of computational experiments with decision tree learning on data
from child-directed speech suggested that such an approach can
successfully learn to classify a disjunction is inclusive or exclusive.
More broadly, cue-based context-dependent mapping is useful for the
acquisition of ambiguous words and interpretations that are consistent
but relatively infrequent in child-directed speech.

\section{Conclusion}\label{conclusion}

\newpage

\section{References}\label{references}

\section{Appendix}\label{appendix}

\subsection{Inter-annotator agreement}\label{inter-annotator-agreement}

\setlength{\parindent}{-0.5in} \setlength{\leftskip}{0.5in}

\hypertarget{refs}{}
\hypertarget{ref-breiman2001random}{}
Breiman, L. (2001). Random forests. \emph{Machine Learning},
\emph{45}(1), 5--32.

\hypertarget{ref-breiman2017classification}{}
Breiman, L. (2017). \emph{Classification and regression trees}. London:
Routledge.

\hypertarget{ref-brown1957linguistic}{}
Brown, R. (1957). Linguistic determinism and the part of speech.
\emph{The Journal of Abnormal and Social Psychology}, \emph{55}(1), 1.

\hypertarget{ref-chierchia2001acquisition}{}
Chierchia, G., Crain, S., Guasti, M. T., Gualmini, A., \& Meroni, L.
(2001). The acquisition of disjunction: Evidence for a grammatical view
of scalar implicatures. In \emph{Proceedings of the 25th Boston
University conference on language development} (pp. 157--168).
Somerville, MA: Cascadilla Press.

\hypertarget{ref-cicchetti1990high}{}
Cicchetti, D. V., \& Feinstein, A. R. (1990). High agreement but low
kappa: II. resolving the paradoxes. \emph{Journal of Clinical
Epidemiology}, \emph{43}(6), 551--558.

\hypertarget{ref-crain2012emergence}{}
Crain, S. (2012). \emph{The emergence of meaning}. Cambridge: Cambridge
University Press.

\hypertarget{ref-demuth2006word}{}
Demuth, K., Culbertson, J., \& Alter, J. (2006). Word-minimality,
epenthesis and coda licensing in the early acquisition of English.
\emph{Language and Speech}, \emph{49}(2), 137--173.

\hypertarget{ref-feinstein1990high}{}
Feinstein, A. R., \& Cicchetti, D. V. (1990). High agreement but low
kappa: I. the problems of two paradoxes. \emph{Journal of Clinical
Epidemiology}, \emph{43}(6), 543--549.

\hypertarget{ref-fisher2010syntactic}{}
Fisher, C., Gertner, Y., Scott, R. M., \& Yuan, S. (2010). Syntactic
bootstrapping. \emph{Wiley Interdisciplinary Reviews: Cognitive
Science}, \emph{1}(2), 143--149.

\hypertarget{ref-fleiss2013statistical}{}
Fleiss, J. L., Levin, B., \& Paik, M. C. (2013). \emph{Statistical
methods for rates and proportions}. New York: John Wiley \&amp; Sons.

\hypertarget{ref-geurts2006exclusive}{}
Geurts, B. (2006). Exclusive disjunction without implicatures.
\emph{Ms., University of Nijmegen}.

\hypertarget{ref-gleitman1990structural}{}
Gleitman, L. (1990). The structural sources of verb meanings.
\emph{Language Acquisition}, \emph{1}(1), 3--55.

\hypertarget{ref-goodman2008does}{}
Goodman, J. C., Dale, P. S., \& Li, P. (2008). Does frequency count?
Parental input and the acquisition of vocabulary. \emph{Journal of Child
Language}, \emph{35}(3), 515--531.

\hypertarget{ref-grice1989studies}{}
Grice, H. P. (1989). \emph{Studies in the way of words}. Cambridge, MA:
Harvard University Press.

\hypertarget{ref-gualmini2000}{}
Gualmini, A., Crain, S., \& Meroni, L. (2000). Acqisition of disjunction
in conditional sentences. In \emph{Proceedings of the boston university
conference on language development}.

\hypertarget{ref-gualmini2000inclusion}{}
Gualmini, A., Meroni, L., \& Crain, S. (2000). The inclusion of
disjunction in child language: Evidence form modal verbs. In
\emph{Proceedings of the North East Linguistic Society 30} (Vol. 30).
Amherst, MA: GLSA.

\hypertarget{ref-haspelmath2007}{}
Haspelmath, M. (2007). Coordination. In T. Shopen (Ed.), \emph{Language
typology and linguistic description,} Cambridge: Cambridge University
Press.

\hypertarget{ref-ho1995random}{}
Ho, T. K. (1995). Random decision forests. In \emph{Proceedings of the
third international conference on document analysis and recognition}
(Vol. 1, pp. 278--282). Washington, DC, USA: IEEE Computer Society.

\hypertarget{ref-kamp1973free}{}
Kamp, H. (1973). Free choice permission. In \emph{Proceedings of the
Aristotelian society} (Vol. 74, pp. 57--74).

\hypertarget{ref-macwhinney2000childes}{}
MacWhinney, B. (2000). \emph{The CHILDES project: The database} (Vol.
2). Mahwah, NJ: Erlbaum.

\hypertarget{ref-morris2008logically}{}
Morris, B. J. (2008). Logically speaking: Evidence for item-based
acquisition of the connectives ``and'' and ``or''. \emph{Journal of
Cognition and Development}, \emph{9}(1), 67--88.

\hypertarget{ref-neisser1962hierarchies}{}
Neisser, U., \& Weene, P. (1962). Hierarchies in concept attainment.
\emph{Journal of Experimental Psychology}, \emph{64}(6), 640.

\hypertarget{ref-notley2012children}{}
Notley, A., Zhou, P., Jensen, B., \& Crain, S. (2012). Children's
interpretation of disjunction in the scope of ``before'': A comparison
of English and Mandarin. \emph{Journal of Child Language},
\emph{39}(03), 482--522.

\hypertarget{ref-pedregosa2011scikit}{}
Pedregosa, F., Varoquaux, G., Gramfort, A., Michel, V., Thirion, B.,
Grisel, O., \ldots{} others. (2011). Scikit-learn: Machine learning in
python. \emph{Journal of Machine Learning Research}, \emph{12}(Oct),
2825--2830.

\hypertarget{ref-pruitt2013interpretation}{}
Pruitt, K., \& Roelofsen, F. (2013). The interpretation of prosody in
disjunctive questions. \emph{Linguistic Inquiry}, \emph{44}(4),
632--650.

\hypertarget{ref-sanchez2018childes}{}
Sanchez, A., Meylan, S., Braginsky, M., MacDonald, K., Yurovsky, D., \&
Frank, M. C. (2018). Childes-db: A flexible and reproducible interface
to the child language data exchange system. PsyArXiv. Retrieved from
\url{psyarxiv.com/93mwx}

\hypertarget{ref-sison1995simultaneous}{}
Sison, C. P., \& Glaz, J. (1995). Simultaneous confidence intervals and
sample size determination for multinomial proportions. \emph{Journal of
the American Statistical Association}, \emph{90}(429), 366--369.

\hypertarget{ref-ubersax2009}{}
Ubersax, J. (2009). Retrieved from
\url{http://www.john-uebersax.com/stat/raw.htm}

\hypertarget{ref-von1968essay}{}
Von Wright, G. H. (1968). An essay in deontic logic and the general
theory of action.






\end{document}
